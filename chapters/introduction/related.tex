\section{Related Work}

\subsection{Consensus Compression}
Nakamoto's original Bitcoin whitepaper~\cite{bitcoin} anticipated the rising
costs of an evergrowing blockchain, and proposed a protocol for lightweight
clients, called ``Simplified Payment Verification'' (SPV). Unlike ``full nodes''
which process and validate the entire blockchain (including all transactions and
signatures), SPV clients only process the proof-of-work chain and transactions
directly pertaining to them. SPV nodes only download the block headers (which in
this thesis we denote $\chain$) of the blockchain and leave out transactions. An
SPV node who has downloaded the chain with the most proof-of-work can validate
the proof-of-work, as the headers are sufficient to do this.
When an SPV client receives a
payment, it requests a Merkle inclusion proof in
order to confirm that the transaction is included in one of the blocks whose
header is already stored by the client. This is possible because the block
header contains a Merkle Tree Root of the transactions.
This gives a significant improvement in communication
complexity, the amount of data that an SPV node needs to download being
$80|\chain|$ for $80$-byte headers in Bitcoin. While today a full node needs to
download more than $250$ GB of data, an SPV node only needs to download $50$ MB
of data. However, as the size of the block has a constant upper bound (currently
about $1$ MB), SPV constitutes a constant improvement over full nodes. In
particular, the amount of data that needs to be downloaded from the network
grows linearly in both the full node and the SPV node cases. They both download
$\bigO(|\chain|)$ data.
Most widely used clients today, among others
mobile wallets based on BitcoinJ, implement SPV.

As an alternative to downloading all block headers at first startup, the SPV
client software could embed a hardcoded checkpoint; blocks prior to which are
ignored.\footnote{See \url{https://bitcoinj.github.io/speeding-up-chain-sync}}
Although this approach is very efficient, it introduces
additional trust assumptions on software maintainers.

The first attempt to compress consensus state to something sublinear without
additional trust assumptions was put
forth in 2012 in a Bitcointalk forum post by Andrew Miller in which he proposes what he
calls the \emph{High-Value-Hash Highway}~\cite{highway}. In that post,
the idea of \emph{superblocks} is introduced and intuition is given about
superchains that capture proof-of-work without presenting it. These ideas give
rise to the first instance of Proofs of Proof-of-Work. Albeit incomplete, some
first ideas are also discussed with regards to the interlink data structure.
A follow up work in 2016 by
Aggelos Kiayias, Nikolaos Lamprou and Aikaterini-Panagiota Stouka formalized
these into a well-defined protocol in their paper
\emph{Proofs of Proofs of Work with Sublinear Complexity}~\cite{popow}. The
paper shows that such protocols can be succinct and presents a concrete protocol
with complexity $\bigO(\log(|\chain|))$. That paper also properly introduces the
superblocks structure, which we make heavy use of in our work. Both the
\emph{highway} forum post and the \emph{PoPoW} paper have been important
inspiration for our work. These ideas form the basis for the creation of
superlight clients.

The protocol defined in these works, the \emph{KLS} protocol, has some
significant shortcomings. First, KLS is \emph{interactive}. This
means that multiple network messages between the prover (a full node peer in the
network) and the verifier (a superlight node or client attempting to
synchronize) are required in order for the superlight node to arrive at a
conclusion about which blockchain is the longest. In these interactions, the
superlight node communicates with multiple provers, at least one of which is
honest, and \emph{interrogates} them by asking questions which are adapted based
on the other proofs that they have received. After sufficient interrogation, the
verifier can draw the correct conclusion. The number of rounds in this
interrogation process can grow to be $\bigO(\log(|\chain|))$. This has impact on
the performance of the superlight client, but also limits the applicability of
the scheme when it comes to applications such as cross-chain certification and
logspace mining. In this thesis, our protocols improve upon theirs to achieve
\emph{non-interactivity} (see Chapter~\ref{chapter:work}), enabling cross-chain
and logspace mining applications in addition to superlight clients (see
Chapter~\ref{chapter:superlight}).

Secondly, while the KLS protocol allows a superlight client to deduce that most
recent $k$ blocks (where $k$ denotes a configurable constant \emph{common prefix
parameter}) of a chain which would be admitted by an honest full node, it gives
no further ability to query the chain for information buried deep within it. In
particular, it does not offer the ability to prove that a transaction took place
in the past, unless $k$ is set to be large. In that case, if
$k \in \Omega(|\chain|)$, the protocol ceases to be succinct. Therefore, the
main application of a superlight client, which involves verifying whether a
transaction took place, is not possible by the KLS protocol, except in limited
circumstances. As such, it constitutes a \emph{suffix proof} protocol, but falls
short of constructing \emph{infix proofs}. In this thesis, we generalize their
construction to allow for the deduction of a quite \emph{generic} class of
predicates about the chain, including old transaction confirmation (see
Chapter~\ref{chapter:work}). Intuitively,
we extend their work to allow any fact about the blockchain which depends on a
polylogarithmic number of blocks to be decided.

Lastly, the security treatment of the KLS construction is incorrect. More
specifically, due to a subtle but critical mistake in the proof of the security
theorem, their conclusion that their protocol achieves security for a $1/2$
adversary with overwhelming probability is false. In fact, there exists a
minority adversary which is able to break security with overwhelming
probability. These issues are explored in this thesis in our development of
\emph{superchain quality}. We subsequently use this property to both build an
attack which we prove works with overwhelming probability against their scheme,
as well as to create a scheme which bypasses the issue for a $1/2$ adversary.
Their initial construction can be proven secure against a $1/3$ adversary,
albeit using a different proof strategy (leveraging our results in
Chapter~\ref{chapter:variable}). We also remark that our schemes are succinct
against all possible adversaries and are not limited to optimistic succinctness.

In addition to treating the above shortcomings, our results generalize in the
more refined model of $\Delta$-bounded delay and variable difficulty. We also
provide experimental results, simulations, concrete security parameters, and
smooth upgrade recommendations.

The practical feasibility of superblock-based constructions and their empirical
analysis has been studied by a series of other
works~\cite{gtklocker,superblocks}.

Our work has inspired a vibrant and challenging line of research by other
groups. Following our NIPoPoW paradigm and bag-of-blocks-based approach, a
different construction named FlyClient~\cite{flyclient} allows the succinct and
secure construction of proofs. Instead of including an interlink vector
which just contains links to the most recent superblock of every level, they
make use of Merkle Mountain Ranges to reference the whole previous blockchain
from every block. This enables them to create a probabilistic proof which is
made non-interactive using the Fiat--Shamir~\cite{fiatshamir} heuristic. The
structure of their scheme is inspired by similar probabilistic
challenge-response protocols that exist in the zero knowledge
setting~\cite{schnorr}. Their protocol is elegant and simple and has seen some
adoption in practice. Additionally, they propose an extension to Merkle
Mountain Ranges, namely Segment Merkle Mountain Ranges which are useful for
variable difficulty constructions (a different approach to ours). While we are
convinced their protocols are secure in principle, contrary to our work in this
thesis, their security analysis so far is only done \emph{in expectation} and
not \emph{with high probability}. A more complete analysis is pending. Lastly,
because the construction of their NIPoPoWs is not superblock-based, but
probabilistic, this means that these proofs are not \emph{online}. Therefore,
if a full node has a proof and mines a block on top of it, they cannot create a
new proof without holding the whole chain. This limitation is inherent in their
construction and means that applications such as logspace mining which rely on
online NIPoPoWs are not possible using their protocol.

A different approach to our superblock-based constructions stems from leveraging
SNARKs and their recursive composition~\cite{snarks,recursive-snarks} as a way
to compress to polylogarithmic size and update a given blockchain.
One such construction is Coda~\cite{coda}. This construction allows the
compression of both proof-of-work and proof-of-stake consensus. A significant
advantage and distinguishing feature of our approach is however the fact that it
does not rely on a common reference string as SNARK-based protocols require, and
so does not require a trusted setup. Moreover, no additional assumptions are
introduced beyond those necessary for the security of the Bitcoin blockchain in
the random oracle model. Lastly, we remark that our approach is simpler in its
construction, as it makes use of only proof-of-work and hash primitives. We
believe this is a great advantage for protocols that will be widely implemented
multiple times. Additionally, it enables a wider audience to audit the
implementation and has a smaller attack surface. We note, however, that our
proof-of-stake construction (in Chapter~\ref{chapter:stake}) is
$\Omega(|\chain|)$, while they achieve the exponentially better result of
$\bigO(\log(|\chain|))$. We only achieve a comparable result in the
proof-of-work case. However, it is unclear whether the Coda construction is
practical. Although the proofs are indeed succinct, the constant factors in
their sizes have not been calculated. As with many zero knowledge schemes, they
can very quickly turn out to be prohibitively large. The practicality of this
scheme remains to be illustrated.

Our proof-of-stake compression scheme makes heavy use of the Ad-Hoc Threshold
Multisignatures primitive, which we introduce in this work. Threshold
multi-signatures were considered before~\cite{pass-asynchronous}, but  without
the ad-hoc characteristic we consider here.

\subsection{Interoperability}
Blockchain interoperability is a problem that has attracted a lot of interest in
the space. Since 2011, the Namecoin~\cite{namecoin} blockchain used the concept
of \emph{merged mining} in which proof-of-work is performed in a 2-for-1 fashion
between the two chains. This mining scheme was later formally
analyzed in 2014~\cite{backbone}. While this does not allow the exchange of
blockchain application data (such as transactions), it constitutes the first
instance of a chain co-existing with another in a symbiotic relationship.

Proposals to allow blockchains to exchange application layer data were first
discussed in 2014 in the work
\emph{Enabling Blockchain Innovations with Pegged Sidechains}~\cite{sidechains}.
This also constitutes the first occurrence of the term \emph{sidechain}. While
they do not propose a concrete mechanism which allows for decentralized
blockchain communication, they describe the need for interoperability among
blockchains. An interesting observation in the work appears in Appendix B in
which the authors discuss the need for secure compact Proofs of Proof-of-Work
(which they call \emph{compact DMMS}) as a prerequisite for the construction of
secure sidechains. The need for a Proofs of Proof-of-Work construction as a
prerequisite for sidechains was pointed out earlier that year by
Friedenbach in the Bitcoin Development mailing list~\cite{friedenbach}. Both of
these mentions refer to superblock-based constructions. These references
highlight that our work in this thesis solves a long-standing open problem with
a multitude of previously known applications.  A significant
contribution of their work is the introduction of the \emph{firewall} property,
which we formalize in this thesis.

These works explore one particular application of sidechains, the ability of a
sidechain to offer smooth software upgrades. They propose that features can be
trialed on a sidechain prior to being adopted in the main chain. Their
definition of a \emph{sidechain} is limited to being a \emph{child} chain of a
particular parent chain; as such, the sidechain's lifecycle fully depends on
that of its parent. In our work, in addition to providing a concrete
implementation of cross-chain proofs which are needed for parent-child-style
sidechains, we generalize the notion of sidechains to allow pre-existing and
stand alone blockchains to communicate as well (see
Chapter~\ref{chapter:sidechains}). Additionally, their exchange of information
is limited to the two-way pegged transfer of value. Our construction allows
conveying \emph{any} information, not just transfers of value. This
distinguishing property becomes more pronounced when applied to smart contract
blockchains in which more generic information of interest can appear. In our
work, a blockchain is a sidechain of another chain if it can generically react
to events on that chain, and so the relationship can be symmetric. The events
which we allow reacting to can be any of the generic predicates that lend
themselves to succinct proofs-of-work or stake.

The first practical implementation of cross-chain information transfer involves
the use of \emph{atomic swaps}. Atomic swaps allow the exchange of two assets
between a party on one chain and a counter-party on another chain atomically.
Atomicity is achieved through Hash Time Locked Contracts~\cite{lightning} and
ensures that either both assets are exchanged, or neither. This makes atomic
swaps useful for decentralized cross-chain trading and makes decentralized
exchanges (DEXes) possible. Atomic swaps were first introduced by Tier
Nolan~\cite{tiernolan}. Multi-asset atomic swaps were studied by
Herlihy~\cite{herlihy2018atomic}. Contrary to atomic swaps, our cross-chain
communication protocols allow the generic transfer of information and are not
limited to asset swapping. Additionally, we do not require a counter-party to
trade with.

One construction which allows the generic transfer of information between chains
without additional trust assumptions is called \emph{BTCRelay}~\cite{btcrelay}.
The premise is to copy all block headers of one chain into the other using a
smart contract. While secure, this structure is not succinct (it is
$\Theta(|\chain|)$). It has been implemented as a smart contract in production
which allows the transfer of information from Bitcoin to Ethereum.

\emph{Drivechains} and \emph{rootstock} are sidechain proposals which require
miners of both chains to be aware of both networks. This follows our definition
of \emph{direct observation} (Chapter~\ref{chapter:sidechains}). While
challenging from an engineering point of view, mutual direct observation makes
the problem theoretically trivial. In practice, mutual direct observation means
that deploying these sidechain schemes on stand-alone chains would require
modifying their miners and full nodes to connect to multiple networks, which
would be unfeasible. In our scheme, miners remain agnostic to the
existence of other chains and connect only to one network.

Plasma~\cite{plasma} (of which Plasma Cash~\cite{plasmacash} is one instance)
creates a child sidechain which is under the control of a centralized custodian
account. The custodian is responsible for creating blocks on the sidechain. The
custodian commits his blocks once every epoch to the parent chain, which
implements a decentralized consensus mechanism. While the custodian is a
centralized entitity, it is not trusted with security, as the construction
allows the participants who hold money on the child sidechain to \emph{abandon
ship} in case foul play is detected. Foul play includes any violation of
\emph{common prefix} on the child sidechain; this means that participants must
wait for the child chain to commit to the parent chain in order to trust that
their transaction has been confirmed. Foul play also includes any
\emph{violation of liveness} in which the custodian fails to create blocks. In
the latter case, the parent chain allows the withdrawal of money from the stale
child chain. One shortcoming of the scheme is that the commitments between the
child chain and the parent chain grow linear in time. However, we believe this
scheme is particularly interesting and practical, as linear cross-communication
will always be necessary between chains that have active cross-chain activity.
While this mechanism cannot connect stand-alone decentralized blockchains, it
can be used to offload some of the burden from an overloaded parent chain,
assisting in scalability.

\emph{Polkadot}~\cite{polkadot}, \emph{Tendermint},
\emph{Cosmos}~\cite{tendermint}, \emph{Liquid} and
\emph{Interledger}~\cite{interledger} also build cross-chain transfers. Their
validation relies on a trusted committees, federations or is left unspecified.
None of the aforementioned constructions include proofs of security. Other
related work includes XCLAIM~\cite{xclaim}, PeaceRelay, COMIT~\cite{comit},
NOCUST~\cite{nocust} and Dogethereum~\cite{dogethereum}.

\todo{summa}
