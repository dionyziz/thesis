\chapter{Conclusion}\label{chapter:conclusion}

Throughout this thesis, we have presented effective consensus compression
mechanisms for all decentralized blockchain protocols, in particular for
both proof-of-work and proof-of-stake. We introduced the NIPoPoWs primitive for
proof-of-work and the ATMS primitive for proof-of-stake. For proof-of-work, we
presented the following variants:

\begin{itemize}
  \item \emph{Charity with goodness} in the static synchronous model, which
        achieves security against a $\frac{1}{2}$ adversary, but succinctness
        only optimistically.
  \item \emph{Charity without goodness} which achieves both security and
        succinctness against a $\frac{1}{3}$ adversary in both the static and
        variable synchronous models and achieves both security and succinctness
        against a $\frac{1}{4}$ adversary in both the static and variable
        $\Delta$-bounded delay model. Succinctness is achieved when difficulty
        does not decrease exponentially.
  \item \emph{Distill}, which achieves comparable results to the above, with the
        additional assumption that difficulty is non-decreasing.
\end{itemize}

For the first among the above, we gave concrete security parameters obtained
through experimental analysis and simulations.

We gave three important applications of our primitives:

\begin{itemize}
  \item \emph{Superlight clients}, which allow the construction of wallets that
        can synchronize faster than standard SPV wallets. The improvement is
        exponential for proof-of-work and constant for proof-of-stake.
  \item \emph{Logarithmic space mining}, which allows the replacement of all
        proof-of-work miners of the protocol with logarithmic-space equivalents
        under the assumption that honest $\frac{1}{4}$ majority holds always.
        The improvement is exponential compared to standard miners for both
        state as well as communication complexity.
  \item \emph{Blockchain interoperability}, which allows any variant of
        blockchains to communicate, namely work/work, work/stake and
        stake/stake. We gave constructions using native support (for stake) as
        well as smart contract-based constructions (for work) and showed how
        they can be used for generic information transfer. Finally, we leveraged
        them to construct one-way and two-way pegs.
\end{itemize}

In addition to improving efficiency of existing solutions, our constructions
hint towards two avenues which are in need of improvement in the blockchain
space more generally. The first avenue concerns \emph{scalability}, a major
current topic of research in the field. While most solutions have been focusing
on layer-2 solutions such as Lightning~\cite{lightning}, our solution of
allowing multiple chains to interoperate and in particular our proposal to
separate the notion of a \emph{cryptocurrency} from its native \emph{blockchain}
allows sidechains to be used to offload transaction traffic off of a main chain
and into multiple sidechains. As long as the majority of transactions remain
within one chain and are not cross-chain transactions, sidechain solutions can
improve the scalability of the main chain. One example means of ensuring sharded
transaction traffic is to create one sidechain per particular industry or
wide geographical location. The second avenue concerns \emph{upgradability} and
the trial of new features. While soft forks and hard forks require consensus
change and may face opposition, sidechains can be used to trial out new features
without requiring all of the main chain to upgrade to these new features. This
can be useful for beta-testing, but also for adopting features that are
considered more risky by the majority. The portion of the population willing to
take the risk can move their capital to a novel sidechain, while the risk-averse
majority can leverage the firewall property to protect their capital on the main
chain.

Overall, our proposals have given rise to vibrant new research directions and
have inspired solutions which are seeing practical adoption across the
cryptocurrency space. Multiple production cryptocurrencies have adopted our
protocols, among others ERGO, nimiq, and WebDollar. Lastly, our primitives have
been implemented and extended by researchers in peer reviewed papers. One
prominent example is FlyClient~\cite{flyclient}, which provides an alternative
implementation to our NIPoPoW primitive.

\input{chapters/conclusion/future}
\section{Epilogue}

Computer science is a data-driven science in which optimization according to
some measurable metric or another always remains the main goal. In our case, we have
optimized the space and communication complexity of blockchain consensus
protocols, and this has given rise to important applications on top. In focusing
on a narrow optimization problem, it is often easy to forget that our work has
moral impact, and one has to keep in mind the moral character of cryptographic
work~\cite{moral}. In addition to the moral dilemmas of secrecy and transparency
faced by our predecessor cryptographers who worked on secure messaging and
digital signatures, as blockchain scientists we are facing broader ethical
questions which stem from the fact that the protocols we design have the
potential for enormous economic and political impact if they are ever to become
mainstream. When I began this thesis four years ago, I was, perhaps na\"ively,
extremely excited about the democratization that blockchain protocols can bring
to the world, from their promise to \emph{bank the unbanked} to the elimination
of the extravagant fees charged by private financial institutions.

Throughout the duration of this work, after studying and understanding the
topics in depth and develping new protocols, some of that initial excitement
faded and turned to partial disillusionment. This came especially through
numerous discussions and research conducted together with my colleague Dimitris
Karakostas and our findings on lack of blockchain
egalitarianism~\cite{egalitarianism} (which does not form part of the present
work). As a new scientist, naturally it is often easy to dismiss legacy systems
such as the existing monetary and banking system by focusing on their
shortcomings instead of their advantages which one often overlooks. Despite more
sober, I am still excited about the future that blockchains and decentralized
protocols can bring if we make good use of them. We shall keep working on them
with ethics in mind. Some big picture questions will keep coming up: Are our new
protocols better than the legacy system, and in which ways? Do they lack in
others? Most importantly, are we building a system which will be a net benefit
to humankind and the less fortunate in our society? Do they preserve or improve
upon egalitarianism and democracy, and in which ways exactly? These are not
exact science questions. While everyone's answers might be different, it is
imperative that we consider the questions and each of us makes their own
judgement. For in solving our mathematical equations and proving our theorems,
we must not forget the real people that our work will impact.

