\section{Discussion}

We have presented a scheme in which full miners are replaced with log\-a\-rith\-mic-space
miners. Our new mining protocol allows miners to only keep storage growing logarithmically
in time. Furthermore, the data communicated to newly bootstrapped nodes is also logarithmic.
We focused on optimizing the \emph{consensus data} portion of blockchains (i.e., block
headers) without concern for the \emph{application data} portion. Our techniques can
be composed with application data optimization techniques.

We have proven our scheme succinct and secure against all $1/3$ adversaries.
Our treatment requires \emph{uninterrupted} honest computational majority
throughout the execution, is in the \emph{static} difficulty model, and works
only for \emph{proof-of-work} blockchains. Let us discuss these
aspects of our construction.

Using our new mathematical framework, the \emph{charity} construction of
Chapter~\ref{chapter:work} (with \emph{certificates of badness} removed) can also be
proven secure against $1/3$ adversaries. However, now that we have the tooling of unsupressibility
available to us, and we can prove the \emph{distill} construction secure and succinct,
we have the additional benefit of the simplicity of comparison at a uniform level.
However, as we will see in Chapter~\ref{chapter:variable}, this \emph{distill} construction
will not work in the variable difficulty setting, and we will have to resort back to the
\emph{charity} construction explored previously.

\noindent
\textbf{Temporary dishonest majority.} One important difference between our scheme
and the existing blockchain protocols is that traditional full nodes are able to verify the \emph{whole}
state evolution of the system from genesis. This allows them to recover in case of
temporary dishonest majority~\cite{dishonest-majority,supremacy}, while our system cannot
do so. Let us consider what could happen in case an adversary temporarily
has the upper hand in a blockchain where everybody is mining using our protocol. Let
$\chain$ denote the chain of the honest parties that has converged. The adversary
begins mining on top of the honest tip. She eventually produces $k+1$ new blocks on
top of $\chain[-1]$, generating an adversarial chain $\chain^*$, prior to the honest
parties advancing by $k+1$ blocks --- a Common Prefix violation. In the block $\chain^*[-k - 1]$,
the adversary places an invalid snapshot; say, a snapshot in which she owns a lot of
money. The rest of the blocks in $\chain^*[-k{:}]$ are filled with valid transactions.
This adversary can then compress this consensus state into a convincing proof,
as state transitions buried $k + 1$ blocks beyond the tip are never checked.
As soon as the honest parties transition to this adversarial chain, the attack
concludes, and no more adversarial supremacy is required. It is critical to understand
what assumptions our protocol mandates: An \emph{uninterrupted} honest majority throughout the
execution. It remains an open question whether it is possible to construct
logarithmic space mining protocols that can withstand temporary adversarial
supremacy.

\noindent
\textbf{Variable difficulty.} We have built and analyzed our logarithmic mining protocol in the
\emph{constant} difficulty setting, i.e., requiring that the target $T$ is a constant.
We strongly suspect, but have not provided proof, that similar protocols to ours work in the
variable difficulty setting. One important change in the protocol that is required
before it can be adapted to variable difficulty settings is that the $\chi$ portion
of the proofs cannot be a constant number of blocks long. Instead, it must be a
suffix which corresponds to \emph{sufficient work having been performed}, the
difficulty of which must correspond to the current target. Simply pruning $k$ blocks
long is insufficient. As such, the verifier
must first gauge the difficulty of the network prior to taking conclusive decisions.
We explore variable difficulty constructions in Chapter~\ref{chapter:variable}.

\noindent
\textbf{Comparison to other NIPoPoWs.} The protocol explored in this chapter
\emph{is} a Non-In\-ter\-active Proof of Proof-of-Work,
akin to the charity NIPoPoWs of the previous chapters, and FlyClient~\cite{flyclient}. Our difference with
FlyClient is the ability to generate \emph{online} proofs, proofs that can be updated as the
blockchain grows. Contrary to our construction, FlyClient requires the sampling of past blocks
to change as new blocks are added to the tip of the blockchain. This is due to their use of the
Fiat--Shamir heuristic~\cite{fiatshamir}. More concretely, a block that was not sampled in the
past may need to be sampled in the future. In our protocol, previously pruned blocks never need
to be salvaged. As \emph{any} block has a potential for future samplability in FlyClient, no
blocks can be discarded, and mining cannot be logarithmic.
We are thus the first to propose a NIPoPoW which is \emph{online}, \emph{succinct}, and
\emph{secure} against all minority ($1/3$) adversaries. All of these are necessary prerequisites
to achieve the desired goal of logarithmic-space mining.
