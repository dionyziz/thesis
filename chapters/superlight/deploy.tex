\section{Deploying with a Soft Fork}\label{sec.deploy}

It should be clear that the protocols proposed in the previous sections can be
deployed in new blockchains from Genesis and can also be deployed in existing
blockchains using a hard fork. It is interesting to consider whether it is also
possible to deploy these protocols using a soft fork. We explore this question
for the concrete case of Ethereum; other cryptocurrencies can be soft forked in
a similar manner.

The protocol of Section~\ref{sec.mining13} against $1/3$ adversaries can easily
be deployed using a soft fork. In order to do that, the interlink set described
in Chapter~\ref{chapter:work} needs to be included in every block. To achieve
backwards compatibility, it can be placed into an \emph{interlink Merkle tree}
whose root is placed in the \texttt{extraData} field. Whenever a
proof is constructed, the prover must prove to the verifier that the proof forms
a chain. Hence, along with each block in the proof, the respective pointer to
the previous block in the proof must be presented. Therefore the prover
accompanies each block with a Merkle Tree proof-of-inclusion which
proves that the pointer is included within the interlink
Merkle tree.

The protocol of Section~\ref{sec.mining12} is more challenging to deploy. Again,
we include the witness-augmented interlink vector into a Merkle tree whose root
is stored in the \texttt{extraData} field.
Recall
that the current block header format of Ethereum is $H(ctr || x || previd)$,
where $ctr$ denotes the mining nonce, $x$ indicates the application data, and $previd$
denotes the hash of the parent block. We require upgraded miners to mine by
searching for a $ctr'$ such that $H(B) \leq T$, where
$B = H(ctr') || x || previd$. The blinded block
value is then defined as $\xi = H(ctr' || H(H(ctr') || x || previd))$ and the
block level is defined as the maximum $\mu$ such that
$\xi \leq \frac{T}{2^\mu}$. In this scheme, the value $H(ctr')$ looks like the
value $ctr$ to unupgraded miners. Note that at this point, it is
indistinguishable for the parties that did not mine the block in question
whether it was mined using the old ($ctr$) or the new protocol ($H(ctr')$) prior
to the value $ctr'$ being disclosed, as, in the old protocol, $ctr$ is chosen
uniformly at random, while in the new protocol, the value $H(ctr')$ is
determined by a Random Oracle. Consider a block $B$ mined by an honest miner
$p$. The value $ctr'$ is revealed after the block $B$ is burried under $2k + 2\ell$
blocks in the view of $p$. The revelation of $ctr'$ takes place by having $p$
create a transaction $tx_\textsc{Reveal}$ which contains
$ctr'$ as well as the index of $B$ in $p$'s chain. As $B$ is stable, this
position will be the same for all other honest parties as well.

While the above scheme works in our cryptographic treatment, it is worth
considering the incentives of such a scheme. In particular, observe that,
without further modification, miners are not incentivized to mine using the new
protocol and can continue mining with the unupgraded protocol, while never
revealing any preimage $ctr'$. Hence, we want to incentivize miners to reveal
the value $ctr'$. For this purpose, we can design the protocol so that the
mining rewards are only paid out conditioned on the fact that $ctr'$ is
properly revealed.

This can be done with a soft fork as follows: The block
beneficiary field is modified to pay out to a smart contract instead of a regular
beneficiary address. The smart contract for this purpose is illustrated in
Algorithm~\ref{alg.blinded-contract}. It is deployed once into
a known address and upgraded miners require that the beneficiary of every block
is the particular smart contract.

 The smart contract is parameterized by two
constant hard-coded parameters $k$ and $\ell$, the common prefix and liveness
parameters respectively. The parameter $k$ ensures that a block buried under $k$
blocks is considered stable and the parameter $\ell$ ensures that after $\ell$
blocks are broadcast during which an honest transaction appears in the mempool,
it will have the chance to be included in the blockchain.

\import{./}{chapters/superlight/algorithms/alg.blinded-contract}

The smart contract is used by an honest miner as follows. The miner places one special
transaction $tx_\textsc{Claim}$ at the beginning of his newly mined block. That transaction calls
the $\textsc{ClaimBlock}$ function of the smart contract and claims the block to
the miner's address before anyone else is able to do so. As the beneficiary
address is taken up by the smart contract's address, this allows the miner to
claim the block with his public key. The miner then waits for $2k + 2\ell$
blocks to pass and subsequently calls the $\textsc{OnTimeReveal}$ function in
the $tx_\textsc{Reveal}$ transaction (and note that if anyone else happens to
call this function, the rightful miner is still paid out). He
passes the block number to be claimed, $idx$, as well as the nonce preimage
$ctr'$ that was used when he mined the block in question and the surrounding
data in the block header $\alpha$ and $\beta$. The smart contract verifies that
the nonce $H(ctr')$ was indeed used by comparing against the block hash, marks
the block reward as paid, and pays the miner. Due to liveness guarantees, the
transaction in which the miner calls $\textsc{OnTimeReveal}$ will be included
from $3k + 2\ell$ blocks until $3k + 3\ell$ blocks after the block in question
was mined and hence will fall within the \textsf{desiredPeriod} mandated by the
smart contract. In case the miner reveals late, the reward is lost.

It is undesirable that the miner reveals $\textsf{ctr'}$ to another party before
$k$ blocks have passed. In case that happens, the party which has received the
revelation can slash the miner's rewards. This is performed as follows. The
party who has early knowledge of $\textsf{ctr'}$ calls the function
$\textsc{EarlyCommit}$ within $k$ blocks of the block in question, committing to
a claim that they will perform an early reveal of the preimage. The commitment
includes the address of the claimant as well as the secret preimage, so that
contesting claimants cannot copy the claim and enter into a transaction race.
This transaction will be included after at most $\ell$ blocks have passed. After it
is included, the claimant waits for $k$ blocks for confirmation to ensure no chain reorg
will cause his commitment to be reversed. Hence, $2k + \ell$ blocks after the
mined block in question, the claimant calls the function $\textsc{EarlyReveal}$
in which an early revelation of the preimage is verified by the smart contract
by ensuring the claim matches the given commitment. In that case, the claimant
is paid a small percentage of the block rewards (in our example we use $10\%$),
while the rest of the rewards are slashed and cannot be later claimed by the
miner. Note that the full amount cannot be paid to the claimant, as we wish to
discourage the miner from being the claimant himself. The transaction making the claim
will take at most $\ell$ blocks to be included, and hence the total period for
early claiming has a duration of $2k + 2\ell$ blocks. The reason we allow
on-time revealing only after this period has passed is to allow for claimants
that have learned the preimage during the first $k$ blocks after the mined block
to have a fair chance of committing and revealing it during the early period.
The honest miner will make the claim only after $3k + 2\ell$ blocks have passed,
so as to avoid the potential for chain reorgs to allow for subsequent early
revelation (once the blockchain has $3k + 2\ell$ blocks, no new blockchain can
be adopted which has different transactions prior to the $2k + 2\ell$ point).

In summary, this scheme disincentivizes miners to reveal prior to seeing that
$k$ blocks have buried their newly mined block (in which case they are
guaranteed to be slashed) and incentivizes them to reveal
after $3k + 2\ell$ blocks have passed (in which case they are guaranteed not to
be slashed), but before $3k + 3\ell$ blocks have passed (so that they still
receive their reward). This is sufficient for the construction of our blinded
mining scheme.

We note here that smart contracts like the one described can be used as soft
forks for any desirable reward distribution change in Ethereum.
