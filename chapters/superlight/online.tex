\section{Mining New Blocks}\label{sec.mining13}

So far, we have used full nodes to help bootstrap newly
booting miners. Can light miners be used to bootstrap newly
booting miners instead? If we can achieve this, then we might
as well get rid of full nodes altogether.

Our light miner already holds a valid proof $\Pi = \pi\chi$
corresponding to an underlying honest full node chain $\chain$ at
the time it is bootstrapped by others. Before further blocks
are mined on the network (either by itself, or by others),
it can send this $\Pi$ to newly booting miners, and they,
too, will be convinced of the current application data snapshot.
The question is how to update this $\Pi$ when a new block is
mined. Suppose a new block $b$ is mined on top of $\chain$,
either by our light miner or by someone else. The underlying
honest chain then becomes $\chain' = \chain b$. Can we
produce a proof $\Pi'$ corresponding to $\chain'$ by only
utilizing $\Pi$? More specifically, given $\Pi = \textsf{Compress}_{m,k}(\chain)$
and $b$, but not given $\chain$, can we produce
$\Pi' = \textsf{Compress}_{m,k}(\chain b)$? Indeed we can. In fact,
it is as simple as evaluating
$\chain'\allowbreak=\allowbreak\textsf{Compress}_{m,k}(\Pi b)$.

\begin{theorem}[Online]
  Consider $\Pi = \textsf{Compress}_{m,k}(\chain)$ generated about an underlying honest chain
  $\chain$, and a block $b$ mined on top of $\chain$. Then
  $\textsf{Compress}_{m,k}(\chain b) = \textsf{Compress}_{m,k}(\Pi b)$.
\end{theorem}
\begin{proof}
  Consider which blocks are sampled and which blocks are pruned
  when calling $\textsf{Compress}_{m,k}(\chain b)$.
  Clearly the block $b$ will be included in both the output of $\textsf{Compress}_{m,k}(\chain b)$
  as well as the output of $\textsf{Compress}_{m,k}(\Pi b)$. All the other blocks selected by
  $\textsf{Compress}_{m,k}(\chain b)$ will already exist in $\Pi$, and in
  the correct positions. For, the blocks selected from a level are the last $2m$
  of a level, or the last $m$ spanning the level above, and adding block $b$
  at the end can only render a previously sampled block irrelevant, but not
  add further block requirements from the past.
\end{proof}

Note also that, when mining a new block $b$, all the data required to compute
the interlink pointers of $b$ is readily available in $\pi\chi$, as $\pi$
contains the most recent $2m$ blocks of every level, and only the most recent
one is needed for interlinking.

\import{./}{chapters/superlight/algorithms/alg.miner.tex}

Our final light miner therefore works as follows. It maintains a \emph{current} proof
$\Pi = \pi\chi$ and mines using $\chi[-1]$ as the chain tip. If it is successful
in mining $b$ on top of $\chi$, it replaces $\Pi$ by setting it to
$\Pi' = \textsf{Compress}_{m,k}(\Pi b)$ and broadcasts this to the network.
As all of the other online miners, light or full, will hold their own $\chi^*$
not differing more than $k$ blocks from $\chi$, it is, in fact, sufficient that it broadcasts
the new $\chi' = \chi[1{:}] b$ portion of $\Pi'$. Now the newly computed $\Pi$ corresponds
to the chain $\chain b$, which the miner never sees, as it has been pruned.
Regardless, $\Pi$ can be used to bootstrap new light miners from genesis.

Consider now the case that our light miner holds a $\Pi = \pi\chi$ and a different
miner mines a new block $b$. By the Common Prefix property, this block will
not deviate more than $k$ blocks from the $\chi$ that our light miner already
holds. Typically, it will be just a block on top of $\chi$, but occassionally
it could correspond to a chain reorganization up to $k$ blocks long. In the case of a reorganization,
the light miner requests the last $k$ blocks $\chi'$ on top of which $b$ was mined.
These can be provided to us if the block $b$ was mined by a light
or a full miner, as both hold and can send $\chi'$.
The blocks in $\chi'$ will intersect the previously known $\chi$ at some
fork point. The light miner checks that the transactions included in this
$\chi'$ can be applied to the application data snapshot that the light
miner has independently calculated for the fork point. This amounts to
full block validation. The light miner also checks that the newly mined
block really does correspond to a longer chain and that a reorganization is warranted
by ensuring that there are more blocks in $\chi'$ after the fork point $b$ than
there are in $\chi$ after the fork point
(i.e., that $|\chi'\{b{:}\}| > |\chi\{b{:}\}|$).
It then replaces the stored proof by setting $\Pi$ to be the proof corresponding
to $\pi\chi$ when the portion of $\chi$ after the most recent common
block between $\chi$ and $\chi'$ is replaced by the blocks in $\chi'$, i.e.,
it updates its stored proof to be
$\Pi' = \textsf{Compress}_{m,k}(\pi \chi\{{:}b\} \chi'\{b{:}\})$.

The light miner is illustrated in Algorithm~\ref{alg.miner}.
At this point, full nodes are no longer necessary. Light miners
can bootstrap from genesis. They have all the data needed to mine on their
own, and to validate newly mined blocks from the network. If a newly booting
light miner wishes to synchronize with the network, they have sufficient
data to help them do so. The blockchain protocol remains exactly the same
as in traditional blockchains, but all the instances of \emph{chains} are
replaced by \emph{proofs} instead. Light miners mine on top of their current
proof instead of mining on top of a chain. When they discover a new block,
they send the newly computed proof instead of a chain.
This concludes our construction.
