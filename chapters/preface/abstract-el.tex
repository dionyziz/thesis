\ifuniversity
\chapter*{ΠΕΡΙΛΗΨΗ}
\thispagestyle{empty}
\else
\begin{center}%
  {\bfseries Περίληψη}%
\end{center}%
\fi

Προτείνουμε τον πρώτο αποκεντρωμένο μηχανισμό διαλειτουργικότητας ανάμεσα σε
αλυσίδες βασισμένες στην απόδειξη εργασίας, στην απόδειξη μεριδίου, ή
συνδυασμούς τους. Για την κατασκευή του, εισάγουμε δύο νέα κρυπτογραφικά
primitives που λειτουργούν ως πιστοποιητικά επικοινωνίας αλυσίδων. Για πηγές
απόδειξης μεριδίου, οι Αυτοτελείς Κατωφλιακές
Πολυ-υπογραφές (ATMS) επιτρέπουν στην απόδειξη ότι το μερίδιο άλλαξε από εποχή
σε εποχή. Για πηγές απόδειξης εργασίας, οι Μη-Διαδραστικές Αποδείξεις Απόδειξης
Εργασίας (NIPoPoWs)
επιτρέπουν την συμπίεση της απόδειξης εργασίας σε σύντομα αλφαριθμητικά που
μειώνουν το μέγεθος μίας αλυσίδας σε μία πολυλογαριθμικού μεγέθους απόδειξη.
Δίνουμε τις πρώτες κατασκευές ATMS και NIPoPoWs. Για την απόδειξη εργασίας,
αποδεικνύουμε ότι οι κατασκευές μας είναι ασφαλείς στο μοντέλο στατικής και
δυναμικής δυσκολίας και πετυχαίνουμε ασφάλεια στο σύγχρονο μοντέλο αλλά και
στο μοντέλο φραγμένων καθυστερήσεων με συγκεκριμένα φράγματα αντιπάλου σε κάθε
περίπτωση. Δίνουμε τον πρώτο ορισμό ασφάλειας πλευρικών αλυσίδων (sidechain) και
αποδεικνύουμε αυστηρά ότι οι κατασκευές μας είναι ασφαλείς. Οι αποδείξεις μας
είναι στο μοντέλο του Bitcoin Backbone για την εργασία και στο Ouroboros για τα
μερίδια.

Τα πιστοποιητικά επικοινωνίας αλυσίδων που προτείνουμε επιτρέπουν την μεταφορά
γενικών πληροφοριών ανάμεσα σε αλυσίδες. Περιγράφουμε πολλαπλές εφαρμογές των
πλευρικών αλυσίδων μας, μεταξύ άλλων μονής κατεύθυνσης μεταφορές χρήματος βασισμένων
στο κάψιμο χρήματος καθώς και διπλής κατεύθυνσης μεταφορές χρήματος.
Εκτός από την διαλειτουργικότητα, τα πρωτόκολλά μας επιτρέπουν την μεταφορά
πληφοροριών που αφορούν μία αλυσίδα για ιδία χρήση.
Αυτό επιτρέπει την κατασκευή υπερελαφρών πορτοφολιών με εκθετικά μικρότερη
επικοινωνιακή πολυπλοκότητα από τα παραδοσιακά πορτοφόλια.
Αυτά τα \emph{υπερελαφριά πορτοφόλια} είναι η πρώτη ασυμπτωτική βελτίωση σε σχέση με
το σύστημα Απλής Πιστοποίησης Πληρωμών (SPV).
Επιπλέον, τα πρωτόκολλά μας μπορούν να χρησιμοποιηθούν στο πλαίσιο της απόδειξης
εργασίας για να δημιουργήσουν \emph{υπερελαφρείς εξορύκτες} οι οποίοι
χρειάζονται μόνο λογαριθμικό χώρο για να εξορύξουν και αποτελούν εκθετική
βελτίωση σε σχέση με κλασικά πρωτόκολλα εξόρυξης απόδειξης εργασίας.
Δείχνουμε την πρακτικότητα των σχημάτων μας με πειράματα, προσομοιώσεις, και
υλοποιήσεις, συμπεριλαμβανωμένων μετρήσεων ασφάλειας και απόδοσης.
Δίνουμε συγκεκριμένες τιμές για τις παραμέτρους ασφαλείας που μπορούν να
χρησιμοποιηθούν στην πράξη. Δουλέψαμε με το χώρο της βιομηχανίας για να
υλοποιηθούν τα σχήματά μας στην πράξη: Τα πρωτόκολλά μας έχουν υλοποιηθεί σε
αληθινές συνθήκες και διαχειρίζονται πραγματικά κεφάλαια στις αλυσίδες απόδειξης
εργασίας ERGO, nimiq, WebDollar και Midnight, καθώς και στην αλυσίδα απόδειξης
μεριδίου Cardano.\\

\ifuniversity
\vfill

{\bfseries ΘΕΜΑΤΙΚΗ ΠΕΡΙΟΧΗ}: Κρυπτογραφία\\

{\bfseries ΛΕΞΕΙΣ ΚΛΕΙΔΙΑ}: Κρυπτογραφία, Blockchains, Διαλειτουργικότητα, Απόδειξη Εργασίας
\clearpage
\fi
