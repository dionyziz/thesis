\ifuniversity
\chapter*{ΣΥΝΟΠΤΙΚΗ ΠΑΡΟΥΣΙΑΣΗ ΤΗΣ ΔΙΔΑΚΤΟΡΙΚΗΣ ΔΙΑΤΡΙΒΗΣ}
\thispagestyle{empty}

Τα αποκεντρωμένα συστήματα βασισμένα στις αλυσίδες απόδειξης εργασίας ή μεριδίου
αναπτύσσονται ραγδιαία από το ντεμπούτο τους μέσω του Bitcoin το 2008 μέχρι σήμερα.
Η ανάπτυξη αυτή έχει οδηγήσει σε δύο κεντρικά προβλήματα στην περιοχή. Από τη μία,
τα συστήματα αλυσίδων που επιβιώνουν μακροχρόνια διατηρούν, αποθηκεύουν και αναμεταδίδουν
στο δίκτυο ολοένα και μακρύτερες αλυσίδες. Ιδιαίτερα σε νεότερα συστήματα όπως το Ethereum
όπου τα blocks των αλυσίδων παράγονται με ταχύτερο ρυθμό αλλά έχουν και μεγαλύτερες κεφαλίδες,
ο συγχρονισμός νέων κόμβων που πρωτοεισέρχονται στο δίκτυο είναι μία αργή και επίπονη υπόθεση.
Από την άλλη, η ανάπτυξη κρυπτονομισμάτων και η ανάγκη για ποικιλομορφία όσο αφορά τα
μοντέλα ασφάλειας, επίδοσης και χαρακτηριστικών στο οικοσύστημα έχει οδηγήσει στην δημιουργία
πλέον εκατοντάδων διαφορετικών κρυπτονομισμάτων τα οποία, εν γένει, αποτελούν απομονωμένους
κόσμους με καμία δυνατότητα επικοινωνίας μεταξύ τους.

Η παρούσα διατριβή δίνει σαφείς και κατηγορηματικές λύσεις στα δύο
παραπάνω προβλήματα, τα οποία σε τεχνικό επίπεδο είναι στενά συνδεδεμένα. Εισάγουμε έναν
τρόπο συμπίεσης της πληροφορίας αμοιβαίας συμφωνίας τόσο σε αλυσίδες απόδειξης εργασίας
όσο και σε απόδειξης μεριδίου. Η συμπίεση αυτή οδηγεί σε πληροφορία που αναπαρίσταται ως
απόδειξη απόδειξης εργασίας ή απόδειξη απόδειξης μεριδίου αντίστοιχα. Τέτοιες αποδείξεις αποδείξεων
είναι εκθετικά μικρότερες από τις πληροφορίες αμοιβαίας συμφωνίας που ανταλλάσσονται σε
παραδοσιακά συστήματα αλυσίδων, δηλαδή τις υποβόσκουσες αποδείξεις εργασίας ή μεριδίου.
Η ανταλλαγή αποδείξεων απόδειξης μπορεί να χρησιμοποιηθεί τόσο για τον τάχιστο
συγχρονισμό υπερελαφρών κόμβων, όσο και για την διαλειτουργικότητα συστημάτων αλυσίδων, στην
οποία περίπτωση οι αποδείξεις αποδείξεων χρησιμοποιούνται ως ένα ασφαλές μεταφορικό μέσο
που φέρει την πληροφορία ώστε αυτή να παραδοθεί σε ένα έξυπνο συμβόλαιο το οποίο μπορεί να
την επιβεβαιώσει.

Εν συντομία, η συμβολή μας είναι διττή: Πρώτον, σχεδιάζουμε ένα πρωτόκολλο το οποίο επιτρέπει
εκθετικά βελτιωμένο συγχρονισμό υπερελαφρών κόμβων με ασφάλεια συγκρίσιμη με αποκεντρωμένους
πλήρεις κόμβους, το πρώτο αποκεντρωμένο αποδεδειγμένα ασφαλές πρωτόκολλο υπερελαφρών κόμβων.
Δεύτερον, σχεδιάζουμε ένα πρωτόκολλο διαλειτουργικότητας ανάμεσα σε αλυσίδες οποιασδήποτε φύσης,
δηλαδή είτε βασισμένων στην απόδειξη εργασίας είτε στην απόδειξη μεριδίου, το οποίο είναι επίσης
αποδεδειγμένα ασφαλές στο αποκεντρωμένο μοντέλο. Είμαστε οι πρώτοι που προτείνουμε ολοκληρωμένη
και αποκεντρωμένη λύση στο πρόβλημα της διαλειτουργικότητας αλυσίδων, ένα πρόβλημα που έχει
τεθεί σχεδόν από την πρώτη στιγμή στην περιοχή των κρυπτονομισμάτων και παρέμενε αναπάντητο
παρ' όλες τις εντατικές και πολλαπλές ανεξάρτητες προσπάθειες της κοινότητας να το αντιμετωπήσει.

\subsection*{Το πρόβλημα του συγχρονισμού}

Όταν ένας κόμβος πρωτοξεκινάει τη λειτουργία του, δηλαδή συνδέεται για πρώτη φορά στο αποκεντρωμένο
δίκτυο ενός συστήματος αλυσίδων, χρειάζεται να συγχρονιστεί με το υπόλοιπο σύστημα. Συγκεκριμένα,
είναι απαραίτητο να αποφανθεί ποια αλυσίδα είναι η προτιμότερη όσο αφορά τα κριτήρια αμοιβαίας
συμφωνίας του συστήματος, δηλαδή η αλυσίδα με την περισσότερη εργασία σε συστήματα απόδειξης εργασίας
ή η αλυσίδα με το μεγαλύτερο μήκος σε συστήματα απόδειξης μεριδίου. Όπως γνωρίζουμε από προηγούμενες
μελέτες~\cite{backbone}, οι κόμβοι δεν καταλήγουν σε \emph{απόλυτη} συμφωνία όσο αφορά αυτές τις αλυσίδες,
αλλά, λόγω της αποκεντρωμένης φύσεως του δικτύου, καταλήγουν ο καθένας σε μία διαφορετική αλυσίδα,
οι οποίες όμως μοιράζονται ένα μακρύ \emph{κοινό πρόθεμα}, δηλαδή το εκάστοτε επιθέματά τους μπορούν
να διαφέρουν κατά το πολύ $k$ blocks, όπου $k$ είναι μία (σταθερή στο χρόνο εκτέλεσης) παράμετρος που
εξαρτάται από την παράμετρο ασφάλειας.

Για να καταλήξει ο εν λόγω κόμβος με ασφάλεια σε μία αλυσίδα η οποία μοιράζεται κοινό πρόθεμα με τους
υπόλοιπους κόμβους είναι απαραίτητο να κατεβάσει αλυσίδες από τους ομότιμους γείτονές του. Στην
πρώτη υλοποίηση πρωτοκόλλων αλυσίδων, οι λεγόμενοι \emph{πλήρεις} κόμβοι κατέβαζαν ολόκληρες τις
αλυσίδες που περιείχαν τόσο
τις κεφαλίδες των blocks όσο και τα περιεχόμενά τους, δηλαδή τις συναλλαγές. Αυτό το πρώτο πρωτόκολλο
σύντομα αποδείχθηκε ότι είχε ασύμφωρη απόδοση, με αποτέλεσμα να υιοθετηθεί το πρωτόκολλο απλοποιημένης
επιβεβαίωσης πληρωμών (SPV) στο οποίο οι λεγόμενοι \emph{ελαφρείς} κόμβοι κατεβάζουν μόνο τις \emph{κεφαλίδες}
των blocks και τις, χειρουργικά επιλεγμένες, συναλλαγές που τους αφορούν. Αυτό το πρωτόκολλο προσφέρει
μία μεγάλη πρακτική βελτίωση στο μέγεθος των δεδομένων που πρέπει να ανταλλαχθούν στο δίκτυο κατά τον
πρώτο συγχρονισμό ενός ελαφρύ κόμβου. Συγκεκριμένα σήμερα, στην περίπτωση για παράδειγμα του Bitcoin, ο
συγχρονισμός ενός πλήρους κόμβου απαιτεί τη μεταφορά $285$ GB, ενώ ένας ελαφρύς κόμβος
απαιτεί τη μεταφορά $38$ MB.

Παρ' όλα αυτά, στο SPV πρωτόκολλο,
ο ρυθμός αύξησης των δεδομένων που πρέπει να ανταλλαχθούν παραμένει \emph{γραμμικός} στο χρόνο εκτέλεσης,
ακριβώς με τον ίδιο τρόπο όπως και στους πλήρεις κόμβους. Αυτό είναι προβληματικό σε δίκτυα που έχουν πολύ ταχύτερους
ρυθμούς παραγωγής blocks, όπως το Ethereum όπου τα blocks παράγονται κάθε $12.5$ δευτερόλεπτα.
Η περίπτωση του Ethereum εκθειάζει το πρόβλημα εντονότερα λόγω των μεγενθυμένων κεφαλίδων blocks σε σχέση
με το Bitcoin. Συγκεκριμένα σήμερα, στο Ethereum ένας πλήρης κόμβος απαιτεί τη μεταφορά $250$ GB, ενώ ένας
ελαφρύς κόμβος απαιτεί τη μεταφορά $4.6$ GB. Είναι σαφές ότι τέτοια μεγέθη είναι μη πρακτικά για πορτοφόλια που
εκτελούνται σε φορητές συσκευές και βρίσκονται πίσω από συνδέσεις διαδικτύου περιορισμένου εύρους ζώνης.
Μία τέτοια χρήση των κρυπτονομισμάτων όμως αποτελεί την πλέον συνήθη χρήση από την οικονομική πλειοψηφία,
καθώς οι περισσότεροι πωλητές και αγοραστές που χρησιμοποιούν κρυπτονομίσματα θα προτιμήσουν να το κάνουν
από την φορητή τους συσκευή και σε φορητές συνθήκες διαδικτύου. Το ερώτημα, λοιπόν, αν είναι εφικτός ο
συγχρονισμός με ανταλλαγή δεδομένων \emph{υπογραμμικών} στο χρόνο εκτέλεσης είναι καίριο, απαραίτητο και
απολύτως κεντρικό για την ευχρηστία και, τελικά, την ευρύτερη υιοθέτηση των κρυπτονομισμάτων.

Στις επόμενες ενότητες θα δούμε ότι, τουλάχιστον στην περίπτωση των αποδείξεων εργασίας,
είναι εφικτός ο σχεδιασμός \emph{υπερελαφρών} κόμβων που απαιτούν
\emph{εκθετικά} λιγότερα δεδομένα για να συγχρονιστούν και συγκεκριμένα δεδομένα που μεγαλώνουν
\emph{λογαριθμικά} με το χρόνο εκτέλεσης. Αυτή η συμπίεση σε \emph{αποδείξεις αποδείξεων εργασίας}
και η στρατηγική σχεδιασμού υπερελαφρών κόμβων αποτελεί κεντρική συμβολή της παρούσας διατριβής.

\subsection*{Η δομή των superblocks}
\subsection*{Αποδείξεις αποδείξεων εργασίας}
\subsection*{Το μοντέλο μεταβλητής δυσκολίας}
\subsection*{Το Δ-φραγμένο μοντέλο}
\subsection*{Αποδείξεις αποδείξεων μεριδίου}
\subsection*{Το πρόβλημα της διαλειτουργικότητας}
\subsection*{Διαλειτουργικότητα με έξυπνα συμβόλαια}

\clearpage
\fi
