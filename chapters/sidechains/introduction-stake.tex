In this section, we formalize the notion of sidechains by proposing a rigorous
cryptographic definition, the first one to the best of our knowledge. The
definition is abstract enough to be able to capture the security for blockchains
based on proof-of-work, proof-of-stake, and other consensus mechanisms.

A critical security feature of a  sidechain system that we formalise
is the \textit{firewall property} in
which a catastrophic failure in one of the chains, such as a violation of its
security assumptions, does not make the other chains vulnerable providing a
sense of limited liability.%
\footnote{To follow the analogy with the term of limited liability in corporate
law, a catastrophic sidechain failure is akin to a corporation going bankrupt
and unable to pay its debtors. In a similar fashion, a sidechain in which the
security assumptions are violated may not be able to cover all of its debtors.
%In fact,
%because it can be under attack,
We give no assurances regarding assets residing on a sidechain if its
security assumptions are broken.
 However, in the same way that stakeholders of a corporation are personally
protected in case of corporate bankruptcy, the mainchain is also protected in
case of sidechain security failures. Our security will guarantee that
each incoming transaction from a sidechain will always have a valid
explanation  in the sidechain ledger
independently of whether the underlying
security assumptions are violated or not. A simple embodiment of this rule
is that a sidechain can return to the mainchain at most as many coins as they have been sent to the sidechain over all time.
}
The   firewall property formalises and generalises the concept of a blockchain \textit{firewall} which was described in high level in~\cite{sidechains}. Informally the blockchain firewall suggests
that no more money can ever return from the sidechain than the amount that was moved
into it. Our general firewall property  allows relying on an
arbitrary definition of exactly how assets can correctly be moved back and forth
between the two chains, we capture this by a so-called
\textit{validity language}. In case of failure, the firewall
ensures that transfers from the sidechain into the mainchain are rejected unless
there exists a (not necessarily unique) plausible history of events on the sidechain that could, in case the
sidechain was still secure, cause the particular transfers to take place.
%If
%the correctness validity language describing the valid transfers between the
%mainchain and the sidechain consists of a simple law of conservation, then the
%limited liability  property captures  precisely a blockchain firewall.

In this section, we outline a concrete exemplary
construction for sidechains for proof-of-stake
blockchains. For conciseness our construction is described
with respect to a generic PoS blockchain consistent with the  Ouroboros
protocol~\cite{ouroboros} that underlies the Cardano blockchain, which is currently one of the largest
pure PoS blockchains by market capitalisation,\footnote{See \url{https://coinmarketcap.com}. }
nevertheless we also discuss how to modify our construction to
operate for
Ouroboros Praos~\cite{praos},
Ouroboros Genesis~\cite{genesis},
Snow White \cite{snowwhite}
and
Algorand \cite{algorand}.

We prove our construction secure   using standard cryptographic
assumptions. We show that our construction (i) supports safe cross-chain value
transfers when the security assumptions of both chains are satisfied, namely
that a majority of honest stake exists in both chains, and (ii) in case of a
one-sided failure,  maintains the firewall property, thus
containing the damage
to the chains whose security conditions have been violated.
%

A critical consideration in a sidechain construction is safeguarding
a new sidechain in its initial ``bootstrapping'' stage against a ``goldfinger'' type of attack \cite{mining-economics,hostile}. Our construction
features a mechanism we call {\em merged-staking}
that allows  mainchain stakeholders who have signalled
sidechain awareness to create sidechain blocks even without moving stake
to the sidechain. In this way, sidechain security can be maintained
assuming honest stake majority among the entities that have signaled sidechain
awareness that, especially in the bootstrapping stage, are expected to be
a large superset of the set of stakeholders that maintain assets
in the sidechain.
%Refel the maintainers
%of the sidechain need to monitor the blocks generated by the mainchain, but the
%maintainers of the mainchain need not be aware of the sidechain's blocks.

Our techniques can be used to facilitate various forms of 2-way peggings
between two chains. As an illustrative example we focus on a parent-child
mainchain-sidechain
configuration where sidechain nodes follow also the mainchain (what we call direct observation) while mainchain nodes need to be able to receive cryptographically
certified signals
from the sidechain maintainers,
taking advantage of the proof-of-stake nature of the underlying protocol. This is achieved by having
mainchain nodes maintain sufficient information about the sidechain that allows
them to authenticate a
small subset of  sidechain stakeholders that is sufficient to
reliably represent the view of a stakeholder majority on the sidechain.
%a recent snapshot of the stakeholder distribution on
%the sidechain (committed in the form of a Merkle-Patricia trie)
This piece of information is updated in regular intervals to account
for  stake shifting on the sidechain.
Exploiting this, each withdrawal  transaction from the sidechain to the mainchain
is signed by this  small subset of sidechain stakeholders.
%that is sufficient to reliably represent the view of a stakeholder majority on the sidechain.
To minimise overheads we batch this authentication information and all the withdrawal transactions from
the sidechain in a single message that will be prepared once per ``epoch.'' We will
refer to this signaling  as
 {\em cross-chain certification}.

In greater detail, adopting some terminology  from \cite{ouroboros},
the sidechain certificate  is constructed by obtaining
signatures  from the set of so-called \emph{slot leaders} of the last
$\Theta(k)$ slots of the previous epoch, where $k$ is the security parameter.
Subsequently, these signatures will be combined together with all necessary
information to convince the mainchain nodes (that do not have access to the
sidechain) that the sidechain certificate is valid.

\noindent {\bf  Related work. }
Sidechains were first proposed as a high level concept in~\cite{sidechains}.
%To the best of
%our knowledge, so far  no
%formal definition of sidechain security exists, and existing constructions are not
%accompanied by formal analysis.
Notable proposed implementations of the concept are given in~\cite{drivechains,lerner}.
In these works, no formal proof of security
is provided and their performance is sometimes akin to maintaining the whole
blockchain within the sidechain, limiting any potential scalability gains.
%
There have been several attempts to create various cross-chain transfer
mechanisms including Polkadot~\cite{polkadot},
Cosmos~\cite{tendermint}, Blockstream's Liquid \cite{federated-interoperability} and Interledger \cite{interledger}. These constructions  differ in various aspects from our work including in that
they focus on proof-of-work or private (Byzantine) blockchains, require
federations, are not decentralized and --- in all cases ---
 lack a formal security model and analysis.
Threshold multi-signatures were considered before, e.g., \cite{pass-asynchronous},
without the ad-hoc characteristic we consider here.
A related primitive that has been considered as potentially useful for enabling
proof-of-work (PoW) sidechains (rather than PoS ones) is a (non-interactive) proof of
proof-of-work~\cite{popow,nipopows}; nevertheless,  these works do not give a
formal security definition for sidechains, nor provide a complete sidechain
construction. We reiterate that while we focus on  PoS, our definitions and model
are fully relevant for the PoW setting as well.

After treating stake-based sidechains, we turn to work-based sidechains in later
sections.
