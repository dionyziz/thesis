\chapter{Background}\label{chapter:background}

This chapter gives an overview of prerequisites upon which we build our
cross-chain protocols.

\section{Notation}
We use standard mathematical notation throughout this work. We define all the
non-standard or unusual notation in this section. We use standard cryptographic
notation, which is also introduced here for reference. The reader unfamiliar
with this notation can consult some reference book in the subject such
as~\cite{katz} for a more complete treatment.

\subsection{Asymptotic probabilistic security}
The shorthand \emph{PPT} is used to denote a probabilistic polynomial-time
Turing machine. In our cryptographic model, we will model adversaries as such.
Our theorems are then by computational reduction, in which we show that bad
events happen only with negligible probability in some security parameter, which
we will denote $\lambda \in \mathbb{N}$. This parameter allows all our
cryptographic primitives to be instantiated with the required level of security;
for example, it provides the number of bits in the output of our hash function.

\begin{definition}[Negligible]
  A function $f: \mathbb{N} \longrightarrow \mathbb{R}^+$ is
  \emph{negligible} if for all $k \in \mathbb{N}$ it holds that
  $f \in \mathcal{O}(\frac{1}{\lambda^k})$.
\end{definition}

We will use the notation $\textsf{negl}$ to denote any negligible function.

\begin{definition}[Overwhelming]
  A function $f: \mathbb{N} \longrightarrow \mathbb{R}^+$ is
  \emph{overwhelming} if it can be written as $f(n) = 1 - \textsf{negl}(n)$ for
  some negligible function $\textsf{negl}$.
\end{definition}

We will define the \emph{security} of various cryptographic protocols by making use of challenger-adversary games in which the challenger is a known Turing Machine defined by us, but the adversary is an \emph{arbitrary} Turing Machine. Herein lies the beauty of cryptography as a science; it allows us to create protocols which we prove our protocols secure against \emph{any} adversary, even those we cannot think of. A protocol will be considered \emph{secure} if no PPT adversary can win the respective game, except with negligible probability. We call such games \emph{unwinnable}.

\begin{definition}[Unwinnability]
  A game defined by a challenger Turing Machine $M_\mathcal{A}(\lambda)$ and parameterized by an adversary $\mathcal{A}$ is \emph{unwinnable} if for all PPT adversaries $\mathcal{A}$ there exists a negligible function $\negl$ such that:

  \[
    \Pr[M_\mathcal{A}(\lambda) = 1] \leq \negl(\lambda)
  \]
\end{definition}

\subsection{Sequences}
We use $[n]$ to denote the set of natural numbers from $0$ up to and including
$n$. We use $\epsilon$ to denote the empty sequence (or empty string). We write
one sequence next to another to denote string concatenation. Likewise, we
concatenate sequences to symbols by juxtaposition.

Our sequences are indexed starting at $0$. Given a sequence $\chain$, we use
$|\chain|$ to denote its length. We use Python notation to denote sequence
addressing. For $i \in [n - 1]$, we denote the $i^\text{th}$ element from the
beginning as $\chain[i]$. The first element of the sequence is thus $\chain[0]$.
For $i \in [n] \setminus \{0\}$, we denote the $i^\text{th}$ element from the
end as $\chain[-i]$. The last element of the sequence is thus $\chain[-1]$. We
call this the \emph{tip} of the sequence. Given $i \in \mathbb{Z},
j \in \mathbb{Z}$ with $i \leq j$, we denote $\chain[i{:}j]$ the subsequence from
$i$ (inclusive) to $j$ (exclusive), that is the sequence which contains exactly
the elements $\chain[i], \chain[i + 1], \dots, \chain[j - 1]$. If $i > j$, then
by convention we set $\chain[i{:}j] = \epsilon$. We allow this \emph{range}
notation to be used with negative indices as well, indicating ranges starting or
ending (or both) in indices considered from the end of the sequence, hence
allowing for $\chain[-i{:}j], \chain[i{:}-j]$, and $\chain[-i{:}-j]$. The left end of
a range can omitted if it is $i = 0$. The right end of a range can be omitted if
it is $j = |\chain|$. For example, $\chain[:-k]$ is the sequence $\chain$ with
the last $k$ elements excluded. In this example, if $|\chain| < k$, then
$\chain[:-k] = \epsilon$.

If an element $A$ is a member of a sequence $\chain$ we will use the notation $A
\in \chain$ to denote this, i.e. that there exist words $w, v$ such that $\chain
= wAv$. It will be clear from the context whether we are speaking about sequence
or sets. Given $A, Z \in \chain$ such that $A$ and $Z$ exist only once in
$\chain$, we denote by $\chain\{A{:}Z\}$ the subsequence of $\chain$ starting from
$A$ (inclusive) and ending in $Z$ (exclusive). If $A = \chain[0]$, it can be
omitted. Omitting $Z$ denotes the sequence starting with $A$ and containing all
subsequent elements until the end of the sequence.

% TODO: \chain\{A:Z\} notation (inclusive, exclusive). But how to include Z?
\section{Cryptographic Primitives}

We now overview the cryptographic primitives we will make use of. In particular,
cryptographically secure hash functions, public-key signatures, and
proof-of-work. Notably, similar to most blockchain protocols, we will not be
using any encryption or decryption. This section is a review. For a full
treatment, refer to any introductory textbook in the subject~\cite{katz}.

\subsection{Hash Functions}
A hash function $H^s: \{0, 1\}^* \longrightarrow \{0, 1\}^\lambda$ is a function
parameterized by the security parameter $\lambda$ which takes any string and
outputs a string of constant size $\lambda$. For technical reasons, the hash
function is instantiated using a key-generating function
$\textsf{Gen}(1^\lambda)$ which generates a hash key $s$.

It is desired that it is difficult to find collisions in hash functions. This is
formalized in the next definition.

\begin{figure}[t]
\begin{algorithm}[H]
    \caption{\label{alg.hash-collision} The collision-finding
             experiment $\textsf{hash-collision}_{\mathcal{A},\Pi}$.}
    \begin{algorithmic}[1]
        \Function{\sf hash-collision$_{\mathcal{A},\Pi}$}{$\lambda$}
            \Let{s}{\textsf{Gen}(1^\lambda)}
            \Let{x_1, x_2}{\mathcal{A}(s)}
            \If{$H^s(x_1) = H^s(x_2) \land x_1 \neq x_2$}
                \State\Return{true}
            \EndIf
            \State\Return{false}
        \EndFunction
        \vskip8pt
    \end{algorithmic}
\end{algorithm}
\end{figure}


\begin{definition}[Collision resistance]
  A hash function $H: \{0, 1\}^* \longrightarrow \{0, 1\}^\lambda$ is called
  \emph{collision resistant} if the game $\textsf{hash-collision}$ is unwinnable.
\end{definition}

\begin{definition}[First pre-image resistance]
  TODO
\end{definition}

\begin{definition}[Second pre-image resistance]
  TODO
\end{definition}

% collision resistance
% SHA256
\subsection{Signatures}
% elliptic curves, secp256k1
\subsection{Proof-of-Work}

\section{The Transaction Layer}
\subsection{Transactions}
\subsection{The UTXO Model}
\subsection{The Account Model}

\section{Authenticated Data Structures}
\subsection{Merkle Trees}
\subsection{Merkle–Patricia Tries}

\section{Blockchains}
\subsection{The Consensus Problem}
\subsection{Blocks}
\subsection{Chains of Blocks}
\subsection{Blockchain Addressing}
\subsection{SPV}

\section{Cryptocurrencies}
\subsection{Bitcoin}
\subsection{Ethereum}

\section{Smart Contracts}
\subsection{Bitcoin Script}
\subsection{Solidity}

\section{Model}
\subsection{The Random Oracle}
% Random Oracles as machines and as random functions
\subsection{Blockchain Backbone}
% Honest Majority Assumption
% Chain Growth
% Common Prefix
% Chain Quality
% The Constant Difficulty assumption and its relaxation
\subsection{The Common Reference String}
% Mention that CRS is not needed (due to Bootstrapping the Blockchain - Directly paper)
\subsection{Ouroboros}
% Epochs

% SPV?
