\subsection{Proof-of-Work}
\cite{pow}\index{Proof-of-Work}
In our decentralized systems in which the number of participants is unknown and
their channels remain unauthenticated, we will assume that the \emph{majority}
of the population is honest~\cite{backbone}. This assumption is summarized in
the following equation:

\begin{definition}[Honest Majority Assumption]\index{Honest Majority}\index{$\delta$}
  We say that a population has \emph{honest majority} if

  \[
  t \leq \leq (1 - \delta)(n - t)
  \]
\end{definition}

This mandates that the number of adversarial parties $t$ is less than the honest
parties $n - t$ by a fraction determined by the parameter $\delta$, the
\emph{honest advantage}\index{Honest Advantage}. We will refer to
this model as the $\frac{1}{2}$-adversary. This assumption is generally
necessary to solve the consensus problem in polynomial time~\cite{okun},
although it can be temporarily relaxed~\cite{dishonest}. Throughout this
work, we will assume that the honest majority holds during \emph{all} rounds.

In some cases, most notably in our constructions of
Chapter~\ref{chapter:variable}, our results will be limited to weaker
adversaries that satisfy more stringent assumptions and are bounded by
$\frac{1}{3}$ or $\frac{1}{4}$ as defined below:

\begin{definition}[Bounded adversaries]
  We say that a population has a $\phi$-bounded adversary (where $\phi$
  typically takes the values of $\frac{1}{2}$, $\frac{1}{3}$, or $\frac{1}{4}$)
  for some $0 < \phi \leq \frac{1}{2}$ if

  \[
  \frac{t}{n - t} \leq (1 - \delta)\phi
  \]
\end{definition}

The honest majority assumption denotes a $\frac{1}{2}$-bounded adversary. If the
number of parties changes from round to round, the respective assumption must be
maintained throughout the whole execution.

While theoretically the honest majority assumption discusses population counts,
in practice this is translated into a majority in computational power. To
achieve this in our model, we bound the number of Random Oracle queries which
are allowed per party per round by some constant $q > 0$, which is the same for all
parties. Limited computational power is then captured by the limited number of
queries to the Random Oracle. As such, each honest party has $q$ available
queries per round, for a total of $(n - t)q$ queries per round, while the
adversary has $tq$ available queries per round. We call this model the
\emph{$q$-bounded model}\index{$q$-bounded Model}.

When working in the synchronous setting, we will set $q$ to be some polynomial
of the security parameter. As the queries are counted \emph{per round}, the
model captures the fact that up to $q$ queries can be made prior to an honest
party being able to communicate the result of their query to the rest of the
honest parties. In the $\Delta$-bounded delay setting, we will simplify by
setting $q = 1$, as the number of queries allowed before a message reaches the
rest of the network can be controlled by adjusting $\Delta$. Concretely, we note
that a $\Delta$-bounded delay setting with $q > 1$ is captured by an equivalent
model in which $\Delta' = \Delta + q$ and $q' = 1$.

The honest parties try to distinguish between messages diffused by the other
honest parties and the adversary. This is a form of \emph{voting}.

Before we get to ordering transactions, let us first observe that the honest
parties that have majority can at least come to a one-bit agreement in
expectation.
