\subsection{Proof-of-Work}
In our decentralized systems in which the number of participants is unknown and
their channels remain unauthenticated, we will assume that the \emph{majority}
of the population is honest~\cite{backbone}. This assumption is summarized in
the following equation:

\glsxtrnewsymbol[description={honest advantage}]{delta}{$\delta$}\glsadd{delta}

\[
t \leq (1 - \delta)(n - t)
\]

This mandates that the number of adversarial parties $t$ is less than the honest
parties $n - t$ by a fraction determined by the parameter $\delta$, the
\emph{honest advantage}\index{Honest Advantage}. We will refer to
this model as the $\frac{1}{2}$-adversary. This assumption is generally
necessary to solve the consensus problem in polynomial time~\cite{okun},
although it can be temporarily relaxed~\cite{dishonest}. Throughout this
work, we will assume that the honest majority holds during \emph{all} rounds.

In some cases, most notably in our constructions of
Chapter~\ref{chapter:variable}, our results will be limited to weaker
adversaries that satisfy more stringent assumptions and are bounded by
$\frac{1}{3}$ or $\frac{1}{4}$ as defined below:

\begin{definition}[Bounded adversaries]
  We say that a population has a $\phi$-bounded adversary (where $\phi$
  typically takes the values of $\frac{1}{2}$, $\frac{1}{3}$, or $\frac{1}{4}$)
  for some $0 < \phi \leq \frac{1}{2}$ if

  \[
  \frac{t}{n} \leq (1 - \delta)\phi
  \]
\end{definition}

The honest majority assumption denotes a $\frac{1}{2}$-bounded adversary. If the
number of parties changes from round to round, the respective assumption must be
maintained throughout the whole execution.

\glsxtrnewsymbol[description={queries per party per round}]{q}{$q$}\glsadd{q}
While theoretically the honest majority assumption discusses population counts,
in practice this is translated into a majority in computational power. To
achieve this in our model, we bound the number of Random Oracle queries which
are allowed per party per round by some constant $q > 0$, which is the same for all
parties. Limited computational power is then captured by the limited number of
queries to the Random Oracle. As such, each honest party has $q$ available
queries per round, for a total of $(n - t)q$ queries per round, while the
adversary has $tq$ available queries per round. We call this model the
\emph{$q$-bounded model}\index{$q$-bounded Model}.

When working in the synchronous setting, we will set $q$ to be some polynomial
of the security parameter. As the queries are counted \emph{per round}, the
model captures the fact that up to $q$ queries can be made prior to an honest
party being able to communicate the result of their query to the rest of the
honest parties. In the $\Delta$-bounded delay setting, we will simplify by
setting $q = 1$, as the number of queries allowed before a message reaches the
rest of the network can be controlled by adjusting $\Delta$. Concretely, we note
that a $\Delta$-bounded delay setting with $q > 1$ is captured by an equivalent
model in which $\Delta' = \Delta + q$ and $q' = 1$.

The honest parties try to distinguish between messages diffused by the other
honest parties and the adversary. This is a form of \emph{voting}. The parties
vote on a message $m$ by solving the \emph{proof-of-work equation}~\cite{pow}
which is defined as follows.

\glsxtrnewsymbol[description={Proof-of-Work target}]{T}{$T$}\glsadd{T}
\begin{definition}[Proof-of-Work]\index{Proof-of-Work}\index{Target}
  Consider a hash function $H: \{0, 1\}^* \rightarrow \{0, 1\}^\kappa$ and some
  $T \leq 2^\kappa$.
  A nonce
  $ctr \in \{0, 1\}^*$ is called \emph{proof-of-work}
  for message $m \in \{0, 1\}^*$ against \emph{target} $T$ if the following inequality
  holds:

  \[
  H(m \concat ctr) \leq T
  \]
\end{definition}

If $H$ is modelled as a Random Oracle,
the best way to solve proof-of-work for a previously unseen message $m$ is by
brute force: iterate through all possible $ctr$ values until a solution is
found.
The message $m$ must have sufficient entropy to ensure no other party is looking
for proof-of-work for the same message. The algorithm that looks for
a proof-of-work solution, given $q$ queries available in a round, is illustrated
in Algorithm~\ref{alg.pow}.

\begin{figure}[t]
\begin{algorithm}[H]
    \caption{\label{alg.pow} The Proof-of-Work discovery algorithm}
    \begin{algorithmic}[1]
      \Function{$\textsf{pow}_{H,T,q}$}{$m$}
          \For{$ctr \gets 1 \text{ to } q$}
              \If{$H(m \concat ctr) \leq T$}
                  \State\Return{$ctr$}
              \EndIf
          \EndFor
          \State\Return{$\bot$}
      \EndFunction
    \end{algorithmic}
\end{algorithm}
\end{figure}


This makes the proof-of-work problem of
finding a nonce a \emph{moderately hard} problem. In the extreme case
where $T = 1$, the problem is computationally hard and lies in $\textsc{EXP}$,
as it takes an exponential number of random oracle queries to find the value
required.
On the other end, if $T = 2^{\kappa-1}$, the problem is computationally easy and
lies in $\textsc{P}$, as only a single query is required to solve the problem
in expectation. When $T$ takes a moderate value, the problem requires a moderate
number of queries until a solution is found. Crucially, the expected number of
queries can be controlled by adjusting the target parameter $T$. Proof-of-work
has the property that it is moderately hard to find, but once found it can be
easily verified by checking that the equation holds.

\begin{definition}[Successful Query]\index{Successful Query}
We call a query to the Random Oracle that satisfies
the proof-of-work equation a \emph{successful query}.
\end{definition}

\glsxtrnewsymbol[description={single query probability of success}]{p}{$p$}\glsadd{p}
\begin{lemma}[Successful Query]
The probability of a new query being successful $p = \frac{T}{2^\kappa}$.
\end{lemma}

We now define the random variables $X_r$, $Y_r$ that specify whether a round
was \emph{successful} and \emph{uniquely successful}.

\glsxtrnewsymbol[description={indicator variable for successful round}]{X}{$X$}\glsadd{X}
\glsxtrnewsymbol[description={indicator variable for uniquely successful round}]{Y}{$Y$}\glsadd{Y}
\glsxtrnewsymbol[description={number of adversarially successful queries}]{Z}{$Z$}\glsadd{Z}
\begin{definition}\index{Successful Round}\index{Uniquely Successful Round}
If an honest party has made a successful query during a round $r$, then we call
$r$ a \emph{successful round} and set $X_r = 1$; otherwise, we set $X_r = 0$.
If among the \emph{honest} queries during $r$ only
\emph{one} was successful, we call $r$ a \emph{uniquely successful round}
and we set $Y_r = 1$; otherwise, we set $Y_r = 0$.
\end{definition}

It is of course possible that, in addition to the honest parties, the adversary
could have had successful queries during a successful or uniquely successful
round. The adversary can also succeed in rounds during which the honest parties
were unsuccessful. We let $Z_{rj} = 1$ if during round $r$ the $j^\text{th}$
adversarial query was successful, where $j$ ranges from $1$ to $tq$; otherwise
we let $Z_{rj} = 0$. For a set of rounds $S$, we define
$X(S) = \sum_{r \in S}X_r$ and $Y(S) = \sum_{r \in S}Y_r$ as well as
$Z(S) = \sum_{r \in S}\sum_{j = 1}^{tq} Z_{rj}$.

In reality, not every honest party has the same computational power, and
multiple honest parties may combine their computational power into a so-called
\emph{mining pool}. These can be captured by treating a more powerful honest
party as multiple honest parties each of which contributes $q$ queries per
round. This is made explicit for the adversary, as she does not incur any
network overhead to achieve communication between the $t$ corrupted parties. On
the contrary, honest players discovering proof-of-work must diffuse it to the
network at a given round and wait for it to be received and validated by the
rest of the honest players at the beginning of the next round (or $\Delta$
rounds later in the $\Delta$-bounded delay model).

These random variables $X, Y, Z$ have the following expected values. For all
rounds $r$:

\begin{itemize}
  \item $\Ex[X_r] = 1 - (1 - p)^{q(n - t)}$
  \item $\Ex[Y_r] = pq(n - t)(1 - p)^{q(n - t - 1)}$
  \item $\Ex[Z_r] = pqt$
\end{itemize}

The random variables $X_r$ and $Y_r$ are Bernoulli distributed, while the
random variable $Z_r$ is Binomially distributed.
The expectation for the adversary assumes that she makes use of all
her available Random Oracle queries. As we will only apply upper bounds to
$\Ex[Z_r]$, this is without loss of generality.

\glsxtrnewsymbol[description={probability of successful round}]{f}{$f$}\glsadd{f}
The expectation $\Ex[X_r]$ is a value critical for our protocols and will be
denoted $f$ throughout this work. If $f$ is too small, the
proof-of-work problem is too hard and there is no progress, harming liveness.
If $f$ is large, the proof-of-work problem is too easy and the honest parties
all solve proof-of-work simultaneously. In that case, there are no intermittent
\emph{periods of silence}\index{Period of Silence} that honest parties can
utilize to reach agreement, and thus persistence is harmed. The parameter $T$
must be carefully calibrated according to $n$ and $q$ so that $f$ takes a
balanced value.

As rounds are independent, the expectations for $\Ex[X(S)]$, $\Ex[Y(S)]$,
$\Ex[Z(S)]$ are equal to $|S|\Ex[X_r]$, $|S|\Ex[Y_r]$, and $|S|\Ex[Z_r]$
respectively. If $|S|$ is sufficiently large, the actual values attained by
$\Ex[X(S)]$, $\Ex[Y(S)]$, and $\Ex[Z(S)]$ will most likely be near their
expectation, except with some small error $\epsilon$. This intuition is formally
captured by the notion of a \emph{typical query distribution}:

\glsxtrnewsymbol[description={Chernoff error}]{epsilon}{$\epsilon$}\glsadd{epsilon}
\begin{definition}[Typical Query Distribution]
  \index{Typical Query Distribution}
  \index{Chernoff Error}\index{Wait Time}
  Let $\epsilon \in (0, 1)$ (the \emph{Chernoff error}) and integer $\lambda
  \geq \frac{2}{f}$ (the \emph{wait time}). Let $S$ be a set of rounds with $|S|
  \geq \lambda$. An execution has \emph{$(\epsilon, \lambda)$-typical query
  distribution} if:

  \begin{itemize}
    \item $(1 - \epsilon)\Ex[X(S)] < X(S) < (1 + \epsilon)\Ex[X(S)]$
    \item $(1 - \epsilon)\Ex[Y(S)] < Y(S)$
    \item $Z(S) < \Ex[Z(S)] + \epsilon\Ex[X(S)]$
  \end{itemize}
\end{definition}

We are now ready to state the full \emph{honest majority assumption}. Our
assumption requires that the honest majority gap $\delta$ is sufficient to
account for both the Chernoff error $\epsilon$ as well as the the lack of
silence due to a potentially large parameter $f$.

\begin{definition}[Honest Majority Assumption]\index{Honest Majority}
  We say that a population has \emph{honest majority} if

  \[
  t \leq (1 - \delta)(n - t)
  \]

  where

  \[
  \delta > 3f + 3\epsilon\,.
  \]
\end{definition}

The honest majority assumption allows us to argue that the number of uniquely
successful queries of the honest parties are generally larger than the number of
successful queries of the adversary. This follows immediately if our query
distribution is typical and is formally proven in the Bitcoin Backbone series of
papers~\cite{backbone,varbackbone}.

Throughout this work, we will be interested in executions in which our
distributions behave nicely. We term these executions
\emph{typical}~\cite{backbone}. The properties we care about pertain to the
values attained by $X, Y, Z$ as well as the collision-resistance and
unpredictability of the Random Oracle.

\begin{definition}[Typical Execution]\index{Typical Execution}
  An execution is \emph{$(\epsilon, \lambda)$-typical} if it has an
  \emph{$(\epsilon, \lambda)$-typical query distribution} and no
  \emph{collisions} or \emph{predictions} have occurred.
\end{definition}

The following theorem is essential for our development and is proven in the
backbone series of papers~\cite{backbone,varbackbone}.

\begin{theorem}[Typicality]
  An execution is $(\epsilon, \lambda)$-typical with overwhelming probability
  in $\epsilon^2 \lambda$.
\end{theorem}

We introduce some additional definitions that are useful for $\Delta$ delays.
In the $\Delta$-bounded delay model, periods of silence that corresponded to
an honest party succeeding in a uniquely successful round now correspond to a
sequence of $\Delta$ consecutive rounds at the beginning of which only one
honest party is successful. The intuitive reason is that the adversary can delay
any other successful query diffusion message within that same $\Delta$ period
and make the two messages coincide in the view of receiving honest parties,
causing disagreement. As such, a useful concept will be to define an
\emph{isolated (uniquely) successful} round as a round that is (uniquely)
successful and is followed by a silence of duration $\Delta$. We also introduce
the respective indicator random variables $X'_i, Y'_i$ for $\Delta$-isolated
(uniquely) successful rounds.

\begin{definition}[$\Delta$-isolated (Uniquely) Successful Round]
\index{$\Delta$-isolated Successful Round}
\glsxtrnewsymbol[description={$\Delta$-isolated successful round indicator}]{X-prime}{$X'$}\glsadd{X-prime}
\glsxtrnewsymbol[description={$\Delta$-isolated uniquely successful round indicator}]{Y-prime}{$Y'$}\glsadd{Y-prime}
A round $r$ is a \emph{$\Delta$-isolated successful round} if $X_r = 1$
and for all $r < r' < r + \Delta$ it holds that $X_{r'} = 0$. In that case we
set $X'_r = 1$; otherwise, we set $X'_r = 0$.

A round $r$ is a \emph{$\Delta$-isolated uniquely successful round} if $Y_r = 1$
and for all $r < r' < r + \Delta$ it holds that $X_{r'} = 0$. In that case we
set $Y'_r = 1$; otherwise, we set $Y'_r = 0$.
\end{definition}

These random variables have the following expectations for every round $r$:

\begin{itemize}
  \item $\Ex[X'_r] = f(1 - f)^{\Delta - 1} \geq f[1 - (\Delta - 1)f]$
  \item $\Ex[Y'_r] \geq f(1 - f)^{2\Delta - 1} \geq f[1 - (2\Delta - 1)f]$
\end{itemize}

In this case, we also have to strengthen the honest majority assumption to
accommodate for the $\Delta$ delay:

\begin{definition}[$\Delta$ Honest Majority]
  We say that a population has \emph{honest majority} with \emph{wait time}
  $\lambda \in \mathbb{N}$ under a $\Delta$-bounded delay with wait if

  \[
  t \leq (1 - \delta)(n - t)
  \]

  where

  \[
  \delta > 5(\epsilon + 2\Delta f + \frac{2 \Delta}{f})\,.
  \]
\end{definition}

Typicality of queries is strengthened as follows:

\begin{definition}[$\Delta$ Typical Query Distribution]
  An execution has \emph{$(\epsilon, \lambda, \Delta)$-typical query
  distribution} with parameters $\epsilon \in (0, 1)$ (the \emph{Chernoff
  error}), integer $\lambda \geq \frac{2}{f}$ (the \emph{wait time}) and integer
  $\Delta$ (the \emph{delay}), if for any set of rounds $S$ with $|S| \geq
  \lambda$ it holds that:

  \begin{itemize}
    \item $(1 - \epsilon)\Ex[X'(S)] < X'(S), X(S) < (1 + \epsilon)\Ex[X(S)]$
    \item $(1 - \epsilon)\Ex[Y'(S)] < Y'(s)$
    \item $Z(S) < \Ex[Z(S)] + \epsilon \Ex[X'(S)]$
  \end{itemize}
\end{definition}

Typical executions are then defined as before and the standard theorem holds:

\begin{theorem}[$\Delta$ Typicality]
  An execution is $(\epsilon, \lambda, \Delta)$-typical with overwhelming
  probability in $\epsilon^2 f^2 (1 - f)^{4\Delta - 2} \lambda$.
\end{theorem}

Before we get to ordering transactions, consider momentarily the classical
problem of a one-bit agreement~\cite{lamport} known as the Byzantine
Agreement\index{Byzantine Agreement} problem. In
the Byzantine Agreement problem, each party boots up with one, potentially
different, \emph{input bit} each, either $0$ or $1$, in its state. The parties
wish to coordinate among each other in order to output a single \emph{output
bit}. The protocol has \emph{agreement} if all honest parties always give output
bits consistent with each other (i.e., if one honest party outputs $b$, all do).
The protocol has \emph{validity} if, when all honest parties are given the same
input bit, the output bit matches the input bit. Agreement or validity alone are
easy to achieve. An example of a protocol that has only agreement asks
all parties to always output $0$. An example of a protocol that has only validity
asks all parties to simply output their input. The challenge is to achieve
both validity and agreement. Contrary to the classical setting, our setting here
is different, because parties are anonymous, unauthenticated, and their count is
unknown. As such, any attempt by the honest parties to report their inputs and
coordinate among each other can be subverted by an adversary who injects
messages that report honest-looking input bits, perhaps differently to each
party, splitting their view.

Instead, proof-of-work must be leveraged. One approach that doesn't work, but
gives intuition about how proof-of-work can be leveraged is as follows.
Initially, each party boots with an input bit $b$. The party repeatedly attempts
to find proof-of-work solutions $ctr_i$ for the message $b \concat r_i$ where $r_i$
is sampled uniformly at random from $\{0, 1\}^\kappa$ to ensure high entropy.
Once some proof-of-work is found, the solution $b \concat r_i \concat ctr_i$ is broadcast
to the network and the process is repeated for as many $r_i$ as possible. After
$\lambda$ rounds have passed, the parties look at the proof-of-work solutions
that have been diffused on the network throughout the execution and count the
votes for bit $0$ and bit $1$. If all honest parties are given the same input,
then they will all produce votes in favour of that input. In \emph{expectation},
the count of these votes will be more than whatever the adversary produces,
thereby achieving validity. However, \emph{agreement} is not achieved because,
if half the honest parties are given input $0$ and the other half are given
input $1$, the adversary can create proof-of-work only in favour of the bit $0$
and hand over that work only to half the parties, causing disagreement. This
lack of agreement hints towards the need to \emph{chain} proof-of-work,
incorporating each previous solution as part of the message for each future
attempt, and continuously work on top of the most proof-of-work chain that
exists on the network, giving rise to \emph{blockchains}. We describe this
process in the next section.
