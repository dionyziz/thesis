\section{Introduction}
Over the past decade, the blockchain space has evolved with hundreds of
different cryptocurrencies and technologies being developed. These systems
are generally isolated and no information can be passed from one to the
other. The critical question this thesis answers is how to move information
from one blockchain to another in a \emph{decentraliazed} manner, i.e.,
without leveraging trusted third parties or groups of third parties.

\section{Dissertation Summary}

\subsection{Problem Statement}
Blockchains are decentralized systems in which a population of maintainers
or \emph{miners} keep track of a series of \emph{transactions} on a public
append-only \emph{ledger}. In Proof-of-Work systems, it is assumed that the
majority of computational power is honest, while in Proof-of-Stake systems,
it is assumed that the majority of the money is in honest hands. Blockchain
nodes maintaining a blockchain do not connect to other blockchains, and,
while it is possible to run complex code on blockchain networks in the form
of smart contracts, these contracts remain isolated and do not have any
networking abilities.

While information from one blockchain can easily be submitted to another,
the critical problem is verifying that the information submitted is
\emph{authentic}, i.e., it corresponds to events that really did take place
in another blockchain.

This problem is closely related to the problem of \emph{light
synchronization}, in which a light wallet booting up with limited state
and memory after being offline for a long time, needs to synchronize
with the rest of the network and verify the validity of transactions.

In both cases, the consensus state of the blockchain, such as the
proof-of-work in work-based blockchains, must be \emph{compressed} so
that it can be submitted for processing either by a smart contract (in
the case of the interoperability problem) or by the light client (in the
case of the synchronization problem). If the compression is
\emph{logarithmic} with respect to the original chain, the protocol
is said to be \emph{succinct}. This is an important property for both
of these problems.

\subsection{Related Work}

The original Bitcoin paper~\cite{bitcoin} includes a mechanism for synchronization
which removes transactions from the picture, but still requires transmission
linear in the blockchain size. Previous attempts at compressing consensus
state are by Miller~\cite{highway} and Kiayias et al.~\cite{popow}.

The problem of interoperability was posited early in the space~\cite{sidechains}.
There have been numerous attempts to interconnect blockchains, but all
have limitations. Atomic swaps~\cite{tiernolan,herlihy2018atomic} allow the swapping of value
between chains, but not the generic exchange of information.
BTCRelay~\cite{btcrelay} is linear in the size of the source chain.
Drivechains~\cite{drivechains} require direct observation for miners
between different chains. Polkadot~\cite{polkadot}, Tendermint, Cosmos~\cite{tendermint},
Liquid~\cite{federated-interoperability}, Interledger~\cite{interledger},
and XCLAIM~\cite{xclaim} all make use of trusted third parties, although with various
trust assumptions and each leveraging different cryptoeconomics.

\subsection{Our Contributions}

\subsubsection{Superblocks and NIPoPoWs}

We create a decentralized blockchain verifier which, having only
\emph{genesis}, connects to multiple provers, at least one of which is honest,
is able to ascertain the confirmation of a transaction. Using its full local
chain, each prover generates a succinct proof and sends it to the verifier.
By comparing the
proofs in terms of the amount of proof-of-work they encode, the verifier deduces
which blockchain contains the most proof-of-work without receiving and
validating every block header. The proofs the provers send are only generated
once and do not require multiple interrogation questions from the verifier. As
such, these proofs are non-interactive and we call them \emph{Non-Interactive
Proofs of Proof-of-Work} (NIPoPoWs).

To create such succinct representations of work, we look at the distribution of
block hashes in the chain. Every valid block $B$ satisfies the proof-of-work
equation $H(B) \leq T$ where $T$ is the mining target, but some blocks satisfy
it better than others. Some blocks so happen to have a hash with value
\emph{much} lower than $T$, even though this is not required for validity and is
not intentional. For example, some blocks will satisfy $H(B) \leq \frac{T}{2}$.
Because the hash function is uniformly distributed, in expectation
\emph{half} the blocks will satisfy $H(B) \leq \frac{T}{2}$, a \emph{quarter} of
them will satisfy $H(B) \leq \frac{T}{4}$, and in general only a $\frac{1}{2^\mu}$ fraction of blocks
will satisfy $H(B) \leq \frac{T}{2^\mu}$. If a block satisfies this inequality
for some $\mu \in \mathbb{N}$, we say that it is of \emph{level} $\mu$ and call
it a $\mu$-superblock.

When a chain $\chain$ is mined, there will therefore exist a subsequence of
it, consisting only of blocks of level $\mu$, which is going to be only
$\frac{|\chain|}{2^\mu}$ blocks long. The core idea of the construction is this:
Instead of sending the full chain, the prover chooses a level $\mu$ and sends
this as a \emph{representative of the underlying work}. Presenting a block of
level $\mu$ captures the fact that work of about $2^\mu$ has happened around it
without presenting that work itself. As such, a superblock is a way for the
prover to \emph{sample} the blockchain in a way that can convince the verifier:
When the verifier receives $m$ blocks of level $\mu$, it can deduce that
approximately $m 2^\mu$ regular blocks must exist around those superblocks.
Constructions based on this simple idea instantiate NIPoPoWs by leveraging
superblocks and we call them \emph{superblock NIPoPoWs}.

We create a different protocol, \emph{Proofs of Proofs-of-Stake}, achieving
similar goals for proof-of-stake blockchains.

\subsubsection{Interoperability}

Given two chains, the \emph{source chain} and the
\emph{destination chain} which are maintained by independent mining populations,
we want to have the destination chain react to an \emph{event} that took place
in the source chain. The miners on one chain are \emph{isolated} from the other,
and they do not connect to its network. However, users interested in the
cross-chain events can connect to both networks. Because users can be
adversarial, the destination chain miners cannot simply trust the users' claim
that an event took place on the source chain -- it needs to be verified.

Events that can be passed from blockchain to blockchain can be virtually any
predicate pertaining to a small amount of data such as a constant number of
transactions or blocks. In practice, these events can be, for example, about the
fact that a transaction with particular metadata took place on the source
blockchain, that a particular amount has been sent or received, that a
particular account holds a certain amount of balance at some point in time, that
some transaction output remains unspent, that a smart contract method was
called with some particular arguments, or that a smart contract's state
variables held certain values at some point in time.

We create support for cross-chain communication for blockchain systems in which
the source blockchain can produce Non-Interactive Proofs of
Proof-of-Work-or-Stake.

The core idea of our sidechain construction is as follows. Whenever an event
takes place within the source blockchain, information about the event as and its
corresponding blockchain is compressed into a Proof of Proof-of-Work (or Stake).
As the proof is a short string, it can be submitted to the destination
blockchain for verification. This verification can be done by the destination
chain miners by
a smart contract. This allows
blockchains that were never designed to be interoperable to communicate.
In this setting, the code for checking the validity of the NIPoPoW as well
as the code for their comparison is written into smart contract format.
This allows the smart contract running in one blockchain to consume NIPoPoWs
generated about other blockchains.

As we can have both proof-of-work and proof-of-stake sources as well as
proof-of-work and proof-of-stake destinations, this construction allows us to
create communication bridges between any combination of consensus mechanisms.
If the information is passed only one-way, then the communication is
unidirectional. This application is still useful as a mechanism to bootstrap a
new cryptocurrency from an old one. We present one way of doing that by
destroying money on the source blockchain and proving that this happened by
providing a relevant proof submitted to the destination chain. On the other
hand, nothing prevents two blockchains from both functioning as a source and a
destination for each other. This allows us to build fully bidirectional
communication channels between blockchains, giving rise to full two-way pegs.

In a two-way peg, the bidirectional communication can be used to move a coin
from one blockchain to another while it retains its nature, decoupling the
notion of a blockchain from that of a cryptocurrency. The lifecycle of the coin
is as follows. Consider two blockchains A and B and an asset which is natively
issued within chain A. Two smart contracts are deployed for interoperability
purposes, one on chain A and one on chain B. Each of the two smart contracts
records the address of its counterpart. Initially, a coin exists in its native
blockchain A. The coin is locked into a special smart contract within the native
blockchain, which ensures that it cannot be further spent. At this stage,
blockchain A functions as a source blockchain for the communication protocol.
The fact that the coin was locked is proven using a Proof of Proof-of-Work or
Stake. This proof
is created by the user that locked her coin. The proof is submitted to a smart
contract which lives on blockchain B. At this stage, blockchain B functions as a
destination chain. The receiving smart contract verifies the Proof of Proof-of-Work
or Stake. At this point, the receiving smart
contract can be certain that the coin was locked into its counterpart which
lives in chain A. Note that the smart contract never directly connected to the
network of chain A. The receiving smart contract mints a new token coin within
blockchain B, equal in nominal value to the value of the locked coin on chain A.
The token on chain B is given the the public key that corresponds to the user
who locked the initial coin on chain A. That token can then be used for exchange
within chain B like a regular currency. However, it is dissimilar from the
native currency of chain B.
Coins can be moved back by using the reverse mechanism.

\subsection{Impact}
We are the first to create both generic decentralized interoperability between
blockchains, and to build superlight clients. Our light client protocols for
proof-of-work are \emph{exponentially} better than previously known solutions.
In the interoperability space, the problem had been long standing almost
since the inception of cryptocurrencies~\cite{sidechains}.

Beyond light clients and interoperability, our protocols can be used in
a multitude of areas that seem unrelated on the surface. As some such examples,
we have leveraged our protocols to
exponentially improve the space requirements for proof-of-work miners,
unilaterally burn cryptocurrency in a provable manner, and to develop financial derivatives
on top of smart contracts in important decentralized finance applications.

Our work has inspired a vibrant and challenging line of research by other
groups. Following our paradigm and definitions, a different
construction named FlyClient~\cite{flyclient} was introduced. Other follow-up
works include Coda~\cite{coda}.

Our protocols have seen wide implementation and deployment in the real world.
Some blockchains implementing our ideas in production are ERGO, nimiq, WebDollar,
Midnight, and Cardano. At the time of writing, they hold tens of billion of
dollars in market capitalization.
