\section{Preliminaries}

Blockchain systems maintain consensus through the dissemination of chains.
A \emph{chain} $\chain$ is a finite sequence of \emph{blocks}, and each
block $B$ is a triplet $(x, s, \textsf{ctr})$. A block id is the cryptographically
secure hash of the triplet $H(x, s, \textsf{ctr})$. Here, $x$ denotes the set of
confirmed messages (transactions), $s$ denotes the block id of the previous
block in the chain, and $\textsf{ctr}$ denotes the \emph{leader election information},
a nonce in the case of proof-of-work systems or a signature in the case of proof-of-stake
systems. The first block in the chain is the genesis block $\mathcal{G}$ in which $s$
is taken to be the empty string $\epsilon$ by convention. We write
$\chain \preccurlyeq \chain'$ if $\chain$ is a (not necessarily strict) prefix of $\chain'$.
We denote by $\chain[i]$ the $i^\text{th}$ block of the chain (zero-based).
We use the Python range notation $\chain[i{:}j]$ to denote the subsequence of blocks,
or \emph{subchain}, from $i$ (inclusive) to $j$ (exclusive).
Omitting an index takes the subchain from the beginning or to the end
respectively.

We denote $\mathcal{E}$ an \emph{execution} of our blockchain
protocol~\cite{EC:GarKiaLeo15,C:GarKiaLeo17}. The execution captures the messages exchanged
by all parties throughout, as well as the random coins produced during the mining process.
We use $\kappa$ to denote the security parameter.

The \emph{block language} $\mathcal{L}_B$ is the set of all syntactically valid blocks and
the \emph{chain language} $\mathcal{L}_\chain \subseteq \mathcal{L}_B^*$ is the set of all
valid chains.

Recall that a blockchain can be upgraded with a \emph{soft fork} or a \emph{hard fork}. In
both cases, the code of the node is modified and the new software is distributed to the users.
Some of the users adopt the new code and those are known as \emph{new} or \emph{upgraded} miners.
The ones that do not upgrade are the \emph{old} or \emph{unupgraded miners} who are running the
old version of the node. Once downloaded, the new code is activated after a designated
\emph{activation block height}. The upgraded software contains new rules that govern the
generation and validation of blocks. In both cases, the old rules are \emph{not} forwards compatible
with the new rules: If an old node generates an old-style block, it will not be validated by
upgraded miners. In the case of a soft fork, the new rules are backwards compatible with the
old rules: If a new node generates a new-style block, it will be validated by old miners.
Provided the upgraded miners constitute a majority, unupgraded nodes will still follow
the longest chain, which will contain only upgraded blocks. Their own blocks will be
rejected, so they will be economically pushed to upgrade their software.
However, in a hard fork, the new rules are \emph{not} backwards compatible with the
old rules: New nodes generate blocks that do not validate according to old rules.
As such, the two populations create two distinct chains after the activation block
height, which constitutes a chain fork. This is sometimes viewed as dangerous.
Nevertheless, even in the case of hard forks, the old population typically
eventually upgrades to the new rules and their temporary fork is abandoned.
