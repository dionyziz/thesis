\documentclass{report}
\usepackage{preamble}

\begin{document}
\begin{titlepage}
  \title{
    {Decentralized Cross-Chain Communication Mechanisms for Proof-of-Work and Proof-of-Stake Blockchains} \\
  {\large National Kapodistrian University of Athens} \\
  }
  \author{Dionysis Zindros}
  \date{\today}
  \maketitle
\end{titlepage}

\newpage

\thispagestyle{empty}
\null

\newpage

\begin{abstract}
During the last decade, the blockchain space has exploded with a plethora of new cryptocurrencies, covering a wide array of different features, performance and security characteristics. Nevertheless, each of these coins functions in a stand-alone manner, independently.  Sidechains have been envisioned as a mechanism to allow blockchains to communicate with one another and, among other applications, allow the transfer of value from one chain to another, but so far there have been no decentralized constructions.

In this thesis, we explore the question of whether it is possible to build interoperability between blockchains and allow them to communicate. Interoperability stands as a stepping stone to enable the solution of problems that have remained important open questions in the blockchain space for years, including interfacing with legacy monetary systems, upgradability, scalability and sustainability. We put forth the first sidechains construction that allows communication between proof-of-stake and proof-of-work blockchains, and combinations thereof, without trusted intermediaries.

In the heart of our sidechains constructions stand two new cryptographic primitives which we introduce in this work. On the proof-of-stake side, the ATMS primitive (Ad-Hoc Threshold Multisignatures) allows attesting to the shifting of stake from epoch to epoch. We provide the first ATMS construction. On the proof-of-work side, the NIPoPoWs primitive (Non-Interactive Proofs of Proof-of-Work) allows compressing proof-of-work into succinct strings that shrink a long blockchain into a single proof. We provide the first NIPoPoWs construction.

We give formal proofs of security for our constructions. In the proof-of-work side, we first explore the setting of constant difficulty and prove our construction secure there. We subsequently analyze our protocol and prove it secure in the variable difficulty setting. In the proof-of-stake side, we require bounded stake-shifting. We assume honest work majority for proof-of-work and honest stake majority for proof-of-stake. Our analysis is in the synchronous setting. Our analysis is cryptographic and our security is proven with overwhelming probability against all polynomial adversaries. For NIPoPoWs, our proof is by reduction to the security of the underlying blockchain and is based on the Bitcoin Backbone model in the Random Oracle model. Our proof-of-work sidechain construction is proven secure by reduction to the security of NIPoPoWs. Our proof-of-stake sidechain construction is studied in the Ouroboros setting and is proven secure by reduction to the security of ATMS.

Our constructions are generic in that we allow the passing of any information between blockchains. Our cross-chain certificates allow proving general predicates about blockchains which can encode any generic events within a blockchain. Thus, our sidechain construction is not limited to the exchange of assets and value.

Our NIPoPoWs and ATMS primitives have more applications beyond sidechains, including the introduction of blockchain clients that have small communication complexity and are non-interactive and are thus superlight. For the proof-of-work case, our proofs are polylogarithmic in the size of the chain. For the proof-of-stake case, our proofs are linear in the number of epochs. We are the first to put forth such superlight client protocols which improve upon the known SPV protocols. Our sidechain constructions have various applications, two prominent examples of which are the ``remote ICO,'' in which an investor pays in currency on one blockchain to receive tokens in another, and the ``two-way peg,'' in which an asset can be transferred from one chain to another and back.

We provide numerous experiments, simulations and implementations illustrating the feasibility of our scheme, including measurements of security and performance metrics and give concrete proposals for the security parameters to be used by our schemes in practice. We demonstrate the feasibility of our construction by providing an implementation in the form of a Solidity smart contract. Our cross-chain protocols have already been implemented in real-world deployments of blockchains by third parties, including ERGO, nimiq, WebDollar for proof-of-work and Cardano for proof-of-stake.

% Future work: we will consider the security of our construction in a semi-synchronous setting. We will also explore the feasibility of various sidechain topologies and the challenges occurring throughout the lifecycle of a blockchain and its sidechains, from the birth of a sidechain where honest majority is difficult to achieve, to the death of a blockchain in which its assets need to be migrated to new blockchains. Last, we will provide full production implementations for both proof-of-stake (in the Cardano cryptocurrency) and in proof-of-work (in the Ethereum, Ethereum Classic and Bitcoin Cash) settings, as well as hybrid settings.
\end{abstract}

\newpage

\tableofcontents

\newpage

\bibliography{bibliography}

\end{document}
